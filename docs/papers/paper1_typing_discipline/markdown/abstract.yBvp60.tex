\begin{abstract}
We present a metatheory of class system design based on
information-theoretic analysis. The three-axis model---(N, B, S) for
Name, Bases, Namespace---induces a lattice of typing disciplines. We
prove that disciplines using more axes strictly dominate those using
fewer (Theorem 2.15: Axis Lattice Dominance).

\textbf{The core contribution is three theorems with universal scope:}

\begin{enumerate}
\def\labelenumi{\arabic{enumi}.}
\item
  \textbf{Theorem 3.13 (Provenance Impossibility --- Universal):} No
  typing discipline over \((N, S)\)---even with access to type
  names---can compute provenance. This is information-theoretically
  impossible: the Bases axis \(B\) is required, and \((N, S)\) does not
  contain it. Not ``our model doesn't have provenance,'' but ``NO model
  without \(B\) can have provenance.''
\item
  \textbf{Theorem 3.19 (Capability Gap = B-Dependent Queries):} The
  capability gap between shape-based and nominal typing is EXACTLY the
  set of queries that require the Bases axis. This is not
  enumerated---it is \textbf{derived} from the mathematical partition of
  query space into shape-respecting and B-dependent queries.
\item
  \textbf{Theorem 3.24 (Duck Typing Lower Bound):} Any algorithm that
  correctly localizes errors in duck-typed systems requires
  \(\Omega(n)\) inspections. Proved by adversary argument---no algorithm
  can do better. Combined with nominal's O(1) bound (Theorem 3.25), the
  complexity gap grows without bound.
\end{enumerate}

These theorems make claims about the universe of possible systems through
three proof techniques: - Theorem 3.13: Information-theoretic impossibility
(input lacks required data) - Theorem 3.19: Mathematical partition
(tertium non datur) - Theorem 3.24: Adversary argument
(lower bound applies to any algorithm)

Additional contributions: - \textbf{Theorem 2.17 (Capability
Completeness):} The capability set
\(\mathcal{C}_B = \{\text{provenance, identity, enumeration, conflict resolution}\}\)
is \textbf{exactly} what the Bases axis provides---proven minimal and
complete. - \textbf{Theorem 8.1 (Mixin Dominance):} Mixins with C3 MRO
strictly dominate object composition for static behavior extension. -
\textbf{Theorem 8.7 (TypeScript Incoherence):} Languages with
inheritance syntax but structural typing exhibit formally-defined type
system incoherence.

All theorems are machine-checked in Lean 4 (2600+ lines, 127
theorems/lemmas, 0 \texttt{sorry} placeholders). Validation
uses 13 case studies from a production bioimage analysis platform
(OpenHCS, 45K LoC Python).

\textbf{Keywords:} typing disciplines, nominal typing, structural
typing, formal methods, class systems, information theory, impossibility
theorems, lower bounds







