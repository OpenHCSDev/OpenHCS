\section{Discussion}\label{discussion}

\subsection{Methodology and Disclosure}\label{methodology-disclosure}

\textbf{Role of LLMs in this work.} This paper was developed through
human-AI collaboration. The author provided the core intuitions,
conjectures, and architectural insights; large language models (Claude,
GPT-4) served as implementation partners---drafting proofs, suggesting
formalizations, and generating code. The Lean 4 proofs were iteratively
refined through this collaboration: the author specified what should be
proved, the LLM proposed proof strategies, and the Lean compiler served
as the ultimate arbiter of correctness.

This methodology aligns with the paper's thesis: the Lean proofs are
\emph{costly signals} (per the companion paper on credibility) because
they require computational verification regardless of how they were
generated. A proof that compiles is correct; the generation method is
epistemically irrelevant to validity. The LLM accelerated exploration
and drafting; the theorems stand or fall on their machine-checked
proofs alone.

\textbf{What the author contributed:} The $(B, S)$ decomposition,
the strict dominance conjecture, the provenance impossibility claim,
the connection to complexity bounds, the case study selection, and the
architectural framing.

\textbf{What LLMs contributed:} LaTeX drafting, Lean tactic suggestions,
literature search assistance, prose refinement, and exploration of proof
strategies.

\textbf{Why this disclosure matters:} Academic norms around authorship
and originality are evolving. We believe transparency about methodology
strengthens rather than weakens the work. The proofs are machine-checked;
the claims are falsifiable; the contribution is the insight, not the
typing.

\subsection{Limitations}\label{limitations}

Our theorems establish necessary conditions for provenance-tracking
systems, but several limitations warrant explicit acknowledgment:

\textbf{Diamond inheritance.} Our theorems assume well-formed MRO
produced by C3 linearization. Pathological diamond inheritance patterns
can break C3 entirely---Python raises \texttt{TypeError} when
linearization fails. Such cases require manual resolution or interface
redesign. Our complexity bounds apply only when C3 succeeds.

\textbf{Runtime overhead.} Provenance tracking stores
\texttt{(value,\ scope\_id,\ source\_type)} tuples for each resolved
field. This introduces memory overhead proportional to the number of
lazy fields. In OpenHCS, this overhead is negligible (\textless{} 1\% of
total memory usage), but systems with millions of configuration objects
may need to consider this cost.

\textbf{Scope: systems where \(B \neq \emptyset\).} Simple scripts where
the entire program fits in working memory may not require provenance
tracking. But provenance is just one of four capabilities (Theorem
2.17). Even without provenance requirements, nominal typing dominates
because it provides identity, enumeration, and conflict resolution at no
additional cost. Our theorems apply universally when
\(B \neq \emptyset\).

\textbf{Python as canonical model.} The formalization uses Python's
\texttt{type(name,\ bases,\ namespace)} because it is the clearest
expression of the two-axis model. This is a strength, not a
limitation: Python's explicit constructor exposes what other languages
obscure with syntax. Table 2.2 demonstrates that 8 major languages
(Java, C\#, Rust, TypeScript, Kotlin, Swift, Scala, C++) are isomorphic
to this model. Theorem 3.50 proves universality.

\textbf{Metaclass complexity.} The \texttt{@global\_pipeline\_config}
chain (Case Study 7) requires understanding five metaprogramming stages:
decorator invocation, metaclass \texttt{\_\_prepare\_\_}, descriptor
\texttt{\_\_set\_name\_\_}, field injection, and type registration. This
complexity is manageable in OpenHCS because it's encapsulated in a
single decorator, but unconstrained metaclass composition can lead to
maintenance challenges.

\textbf{Lean proofs assume well-formedness.} Our Lean 4 verification
includes \texttt{Registry.wellFormed} and MRO monotonicity as axioms
rather than derived properties. We prove theorems \emph{given} these
axioms, but do not prove the axioms themselves from more primitive
foundations. This is standard practice in mechanized verification (e.g.,
CompCert assumes well-typed input), but limits the scope of our
machine-checked guarantees.

\textbf{Validation scope.} The formal results (Theorems 3.5,
3.13, Corollary 6.3) are proven universally for any system where
\(B \neq \emptyset\). These proofs establish \emph{what is impossible}:
provenance cannot be computed without the bases axis
(information-theoretically impossible, not merely difficult). The
case studies (Section 5) demonstrate these theorems in a production
codebase. The \emph{direction} of the claims---that capability gaps
translate to error reduction---follows from the formalism: if provenance is
impossible without nominal typing (Corollary 6.3), and provenance is
required (PC = 1), then errors \emph{must} occur under duck typing. The
\emph{magnitude} of the effect is codebase-specific; the
\emph{existence} of the effect is not. We distinguish:

\begin{itemize}
\tightlist
\item
  \textbf{Universal (proven):} Capability gap exists, provenance is
  impossible under duck typing, nominal typing strictly dominates.
\item
  \textbf{Singular (observed):} 47 \texttt{hasattr()} calls eliminated,
  centralized error detection via ABC contracts.
\end{itemize}

We call for replication studies on other codebases to measure the
magnitude of the effect across different architectural patterns. The
formal results predict that \emph{some} positive effect will be
observed in any \(B \neq \emptyset\) system requiring provenance; the
specific multipliers are empirical questions.

\paragraph{8.1.1 Axiom Methodology}\label{axiom-methodology}

\textbf{Theorem 8.1a (Axiom Scope).} The axioms
\texttt{Registry.wellFormed} and MRO monotonicity are \emph{descriptive}
of well-formed programs, not \emph{restrictive} of the proof's scope.
Programs violating these axioms are rejected by the language runtime
before execution.

\emph{Proof.} We enumerate each axiom and its enforcement:

\small
\begin{longtable}[]{@{}
  >{\raggedright\arraybackslash}p{(\linewidth - 4\tabcolsep) * \real{0.1522}}
  >{\raggedright\arraybackslash}p{(\linewidth - 4\tabcolsep) * \real{0.3913}}
  >{\raggedright\arraybackslash}p{(\linewidth - 4\tabcolsep) * \real{0.4565}}@{}}
\toprule\noalign{}
\begin{minipage}[b]{\linewidth}\raggedright
Axiom
\end{minipage} & \begin{minipage}[b]{\linewidth}\raggedright
What It Requires
\end{minipage} & \begin{minipage}[b]{\linewidth}\raggedright
Language Enforcement
\end{minipage} \\
\midrule\noalign{}
\endhead
\bottomrule\noalign{}
\endlastfoot
\texttt{Registry.wellFormed} & No duplicate ABC registrations, no cycles
& \texttt{ABCMeta.register()} raises on duplicates; Python rejects
cyclic inheritance \\
MRO monotonicity & If A \textless: B, A precedes B in MRO & C3
linearization guarantees this; violation raises \texttt{TypeError} at
class definition \\
MRO totality & Every class has a linearizable MRO & C3 fails for
unlinearizable diamonds; \texttt{TypeError} at class definition \\
\texttt{isinstance} correctness & \texttt{isinstance(x,\ T)} iff
\texttt{type(x)} in T's subclass set & Definitional in Python's data
model \\
\end{longtable}

A program violating any of these axioms fails at class definition time
with \texttt{TypeError}. Such a program is not a valid Python
program---it cannot be executed. Therefore, our theorems apply to
\emph{all valid programs}. \(\blacksquare\)

\textbf{Corollary 8.1b (Axiom Scope).} A claim that the axioms are too
strong would require exhibiting: 1. A valid, executable Python program
where the axioms fail, AND 2. A scenario where this program requires
typing discipline analysis.

Programs where axioms fail are not valid programs---they crash at
definition time. The axioms characterize well-formed programs, which is
the standard scope for type system analysis.

\textbf{Comparison to prior art.} This methodology is standard in
mechanized verification: - \textbf{CompCert} (verified C compiler):
Assumes input is well-typed C - \textbf{seL4} (verified microkernel):
Assumes hardware behaves according to spec - \textbf{CakeML} (verified
ML compiler): Assumes input parses successfully

We follow the same pattern: assume the input is a valid program
(accepted by Python's runtime), prove properties of that program.
Proving that Python's parser and class system are correct is out of
scope---and unnecessary, as Python's semantics are the \emph{definition}
of what we're modeling.

\subsection{The Typing Discipline
Hierarchy}\label{the-typing-discipline-hierarchy}

Theorem 2.10d establishes that duck typing is incoherent. Theorem 2.10g
establishes that structural typing is eliminable when
\(B \neq \emptyset\). Together, these results collapse the space of
valid typing disciplines.

\textbf{The complete hierarchy:}

\small
\begin{longtable}[]{@{}
  >{\raggedright\arraybackslash}p{(\linewidth - 6\tabcolsep) * \real{0.2500}}
  >{\raggedright\arraybackslash}p{(\linewidth - 6\tabcolsep) * \real{0.2292}}
  >{\raggedright\arraybackslash}p{(\linewidth - 6\tabcolsep) * \real{0.2708}}
  >{\raggedright\arraybackslash}p{(\linewidth - 6\tabcolsep) * \real{0.2500}}@{}}
\toprule\noalign{}
\begin{minipage}[b]{\linewidth}\raggedright
Discipline
\end{minipage} & \begin{minipage}[b]{\linewidth}\raggedright
Coherent?
\end{minipage} & \begin{minipage}[b]{\linewidth}\raggedright
Eliminable?
\end{minipage} & \begin{minipage}[b]{\linewidth}\raggedright
When Valid
\end{minipage} \\
\midrule\noalign{}
\endhead
\bottomrule\noalign{}
\endlastfoot
Duck typing (\(\{S\}\)) & No (Thm 2.10d) & N/A & Never \\
Structural (\(\{N, S\}\)) & Yes & Yes, when \(B \neq \emptyset\) (Thm
2.10g) & Only when \(B = \emptyset\) \\
Nominal (\(\{N, B, S\}\)) & Yes & No & Always (when
\(B \neq \emptyset\)) \\
\end{longtable}

\textbf{Duck typing} is incoherent: no declared interface, no complete
compatibility predicate, no position on structure-semantics
relationship. This is never valid.

\textbf{Structural typing (Protocol)} is coherent but eliminable: for
any system using Protocol at boundaries, there exists an equivalent
system using nominal typing with explicit adapters (Theorem 2.10g). The
only ``value'' of Protocol is avoiding the 2-line adapter class.
Convenience is not a capability.

\textbf{Nominal typing (ABC)} is coherent and non-eliminable: it is the
only necessary discipline for systems with inheritance.

\textbf{The eliminability argument.} When integrating third-party type
\(T\) that cannot inherit from your ABC:

\begin{verbatim}
\# Structural approach (Protocol) {- implicit}
@runtime\_checkable
class Configurable(Protocol):
    def validate(self) {-\textgreater{}} bool: ...

isinstance(their\_obj, Configurable)  \# Hope methods match
\end{verbatim}

\begin{verbatim}
\# Nominal approach (Adapter) {- explicit}
class TheirTypeAdapter(TheirType, ConfigurableABC):
    pass  \# 2 lines. Now in your hierarchy.

adapted = TheirTypeAdapter(their\_obj)  \# Explicit boundary
isinstance(adapted, ConfigurableABC)   \# Nominal check
\end{verbatim}

The adapter approach is strictly more explicit. ``Explicit is better
than implicit'' (Zen of Python). Protocol's only advantage---avoiding
the adapter---is a convenience, not a typing capability.

\textbf{Languages without inheritance.} Go's struct types have
\(B = \emptyset\) by design. Structural typing with declared interfaces
is the only coherent option. Go does not use duck typing; Go interfaces
are declared \cite{goSpec}. This is why Go's type system is sound
despite lacking inheritance.

\textbf{The final collapse.} For languages with inheritance
(\(B \neq \emptyset\)): - Duck typing: incoherent, never valid -
Structural typing: coherent but eliminable, valid only as convenience -
Nominal typing: coherent and necessary

The only \emph{necessary} typing discipline is nominal. Everything else
is either incoherent (duck typing) or reducible to nominal with trivial
adapters (structural typing).

\subsection{Future Work}\label{future-work}

\textbf{Gradual nominal/structural typing.} TypeScript supports both
nominal (via branding) and structural typing in the same program.
Formalizing the interaction between these disciplines, and proving
soundness of gradual migration, would enable principled adoption
strategies.

\textbf{Trait systems.} Rust traits and Scala traits provide multiple
inheritance of behavior without nominal base classes. Our theorems apply
to Python's MRO, but trait resolution uses different algorithms.
Extending our complexity bounds to trait systems would broaden
applicability.

\textbf{Automated complexity inference.} Given a type system
specification, can we automatically compute whether error localization
is O(1) or \(\Omega\)(n)? Such a tool would help language designers
evaluate typing discipline tradeoffs during language design.

\subsection{Implications for Language
Design}\label{implications-for-language-design}

Language designers face a fundamental choice: provide nominal typing
(enabling provenance), structural typing (for \(B = \emptyset\)
boundaries), or both. Our theorems inform this decision:

\textbf{Provide both mechanisms.} Languages like TypeScript demonstrate
that nominal and structural typing can coexist. TypeScript's
``branding'' idiom (using private fields to create nominal distinctions)
validates our thesis: programmers need nominal identity even in
structurally-typed languages. Python provides both ABCs (nominal) and
\texttt{Protocol} (structural). Our theorems clarify the relationship:
when \(B \neq \emptyset\), nominal typing (ABCs) strictly dominates
Protocol (Theorem 2.10j). Protocol provides convenience (avoiding adapters)
but this is not a capability---ABCs can also integrate external types via
adapters. Protocol is dominated: it provides a strict subset of capabilities.

\textbf{MRO-based resolution is near-optimal.} Python's descriptor
protocol combined with C3 linearization achieves O(1) field resolution
while preserving provenance. Languages designing new metaobject
protocols should consider whether they can match this complexity bound.

\textbf{Explicit \texttt{bases} makes nominal typing strictly optimal.} If a language
exposes explicit inheritance declarations (\texttt{class\ C(Base)}),
Theorem 3.4 (Nominal Pareto-Dominance) applies: nominal typing strictly
dominates structural typing. Language designers cannot add inheritance
to a structurally-typed language without creating capability gaps that
nominal typing would eliminate.

\subsection{Derivable Code Quality
Metrics}\label{derivable-code-quality-metrics}

The formal model yields four measurable metrics that can be computed
statically from source code:

\textbf{Metric 1: Duck Typing Density (DTD)}

\begin{verbatim}
DTD = hasattr_calls / KLOC
\end{verbatim}

Measures ad-hoc capability probing. High DTD where \(B \neq \emptyset\)
indicates discipline violation. We count only \texttt{hasattr()}, not
\texttt{getattr()} or \texttt{try/except AttributeError}, because
\texttt{hasattr()} is specifically capability detection (``does this
object have this attribute?'')---the operational signature of duck
typing (Definition 2.10c). \texttt{getattr()} without a fallback is
explicit attribute access; \texttt{getattr()} with a fallback or
\texttt{try/except AttributeError} may indicate duck typing but also
appear in legitimate metaprogramming (descriptors, \texttt{\_\_getattr\_\_}
hooks, optional feature detection at system boundaries). The theorem
backing (Theorem 2.10d) establishes \texttt{hasattr()} as the incoherent
probe; other patterns require case-by-case analysis.

\textbf{Metric 2: Nominal Typing Ratio (NTR)}

\begin{verbatim}
NTR = (isinstance_calls + type_as_dict_key + abc_registrations) / KLOC
\end{verbatim}

Measures explicit type contracts. High NTR indicates intentional use of
inheritance hierarchy.

\textbf{Metric 3: Provenance Capability (PC)} Binary metric: does the
codebase contain queries of the form ``which type provided this value''?
Presence of \texttt{(value,\ scope,\ source\_type)} tuples, MRO
traversal for resolution, or \texttt{type(obj).\_\_mro\_\_} inspection
indicates PC = 1. If PC = 1, nominal typing is mandatory (Corollary
6.3).

\textbf{Metric 4: Resolution Determinism (RD)}

\begin{verbatim}
RD = mro_based_dispatch / (mro_based_dispatch + runtime_probing_dispatch)
\end{verbatim}

Measures O(1) vs \(\Omega\)(n) error localization. RD = 1 indicates all
dispatch is MRO-based (nominal). RD = 0 indicates all dispatch is
runtime probing (duck).

\textbf{Tool implications:} These metrics enable automated linters. A
linter could flag \texttt{hasattr()} in any code where
\(B \neq \emptyset\) (DTD violation), suggest \texttt{isinstance()}
replacements, and verify that provenance-tracking codebases maintain NTR
above a threshold.

\textbf{Empirical application:} In OpenHCS, DTD dropped from 47 calls in
the UI layer (before PR \#44) to 0 after migration. NTR increased
correspondingly. PC = 1 throughout (dual-axis resolver requires
provenance). RD = 1 (all dispatch is MRO-based).

\subsection{Hybrid Systems and Methodology
Scope}\label{hybrid-systems-and-methodology-scope}

Our theorems establish necessary conditions for provenance-tracking
systems. This section clarifies when the methodology applies and when
shape-based typing is an acceptable concession.

\paragraph{8.6.1 Structural Typing Is Eliminable (Theorem
2.10g)}\label{structural-typing-is-eliminable-theorem-2.10g}

\textbf{Critical update:} Per Theorem 2.10g, structural typing is
\emph{eliminable} when \(B \neq \emptyset\). The scenarios below
describe when Protocol is \emph{convenient}, not when it is
\emph{necessary}. In all cases, the explicit adapter approach (Section
8.2) is available and strictly more explicit.

\textbf{Retrofit scenarios.} When integrating independently developed
components that share no common base classes, you cannot mandate
inheritance directly. However, you \emph{can} wrap at the boundary:
\texttt{class\ TheirTypeAdapter(TheirType,\ YourABC):\ pass}. Protocol
is a convenience that avoids this 2-line adapter. Duck typing is never
acceptable.

\textbf{Language boundaries.} Calling from Python into C libraries,
where inheritance relationships are unavailable. The C struct has no
\texttt{bases} axis. You can still wrap at ingestion: create a Python
adapter class that inherits from your ABC and delegates to the C struct.
Protocol avoids this wrapper but does not provide capabilities the
wrapper lacks.

\textbf{Versioning and compatibility.} When newer code must accept older
types that predate a base class introduction, you can create versioned
adapters:
\texttt{class\ V1ConfigAdapter(V1Config,\ ConfigBaseV2):\ pass}.
Protocol avoids this but does not provide additional capabilities.

\textbf{Type-level programming without runtime overhead.} TypeScript's
structural typing enables type checking at compile time without runtime
cost. For TypeScript code that never uses \texttt{instanceof} or class
identity (effectively \(B = \emptyset\) at runtime), structural typing
has no capability gap because there's no \(B\) to lose. However, see
Section 8.7 for why TypeScript's \emph{class-based} structural typing
creates tension---once you have \texttt{class\ extends}, you have
\(B \neq \emptyset\).

\textbf{Summary.} In all scenarios with \(B \neq \emptyset\), the
adapter approach is available. Protocol's only advantage is avoiding the
adapter. Avoiding the adapter is a convenience, not a typing capability
(Corollary 2.10h).

\paragraph{\texorpdfstring{8.6.2 The \(B \neq \emptyset\) vs
\(B = \emptyset\)
Criterion}{8.6.2 The B \backslash neq \backslash emptyset vs B = \backslash emptyset Criterion}}\label{the-b-neq-emptyset-vs-b-emptyset-criterion}

The only relevant question is whether inheritance exists:

\textbf{\(B \neq \emptyset\) (inheritance exists):} Nominal typing is
correct. Adapters handle external types (Theorem 2.10j). Examples: -
OpenHCS config hierarchy:
\texttt{class\ PathPlanningConfig(GlobalConfigBase)} - External library
types: wrap with
\texttt{class\ TheirTypeAdapter(TheirType,\ YourABC):\ pass}

\textbf{\(B = \emptyset\) (no inheritance):} Structural typing is the
only option. Examples: - JSON objects from external APIs - Go interfaces
- C structs via FFI

The ``greenfield vs retrofit'' framing is obsolete (see Remark after
Theorem 3.62).

\paragraph{8.6.3 System Boundaries}\label{system-boundaries}

Systems have \(B \neq \emptyset\) components (internal hierarchies) and
\(B = \emptyset\) boundaries (external data):

\begin{verbatim}
# B != {}: internal config hierarchy (use nominal)
class ConfigBase(ABC):
    @abstractmethod
    def validate(self) -> bool: pass

class PathPlanningConfig(ConfigBase):
    well_filter: Optional[str]

# B = {}: parse external JSON (structural is only option)
def load_config_from_json(json_dict: Dict[str, Any]) -> ConfigBase:
    # JSON has no inheritance—structural validation at boundary
    if "well_filter" in json_dict:
        return PathPlanningConfig(**json_dict)  # Returns nominal type
    raise ValueError("Invalid config")
\end{verbatim}

The JSON parsing layer is \(B = \emptyset\) (JSON has no inheritance).
The return value is \(B \neq \emptyset\) (ConfigBase hierarchy). This is
correct: structural at data boundaries where \(B = \emptyset\), nominal
everywhere else.

\paragraph{8.6.4 Scope Summary}\label{scope-summary}

\small
\begin{longtable}[]{@{}
  >{\raggedright\arraybackslash}p{(\linewidth - 4\tabcolsep) * \real{0.2143}}
  >{\raggedright\arraybackslash}p{(\linewidth - 4\tabcolsep) * \real{0.4286}}
  >{\raggedright\arraybackslash}p{(\linewidth - 4\tabcolsep) * \real{0.3571}}@{}}
\toprule\noalign{}
\begin{minipage}[b]{\linewidth}\raggedright
Context
\end{minipage} & \begin{minipage}[b]{\linewidth}\raggedright
Typing Discipline
\end{minipage} & \begin{minipage}[b]{\linewidth}\raggedright
Justification
\end{minipage} \\
\midrule\noalign{}
\endhead
\bottomrule\noalign{}
\endlastfoot
\(B \neq \emptyset\) (any language with inheritance) & Nominal
(mandatory) & Theorem 2.18 (strict dominance), Theorem 2.10j (adapters
dominate Protocol) \\
\(B = \emptyset\) (Go, JSON, pure structs) & Structural (correct) &
Theorem 3.1 (namespace-only) \\
Language boundaries (C/FFI) & Structural (mandatory) & No inheritance
available (\(B = \emptyset\) at boundary) \\
\end{longtable}

\textbf{Removed rows:} - ``Retrofit / external types $\rightarrow$ Structural
(acceptable)'' --- Adapters exist (Theorem 2.10j); structural is dominated. -
``Small scripts / prototypes $\rightarrow$ Duck (acceptable)'' ---
Duck typing is incoherent for B-dependent queries (Theorem 2.10d).

The methodology states: \textbf{if \(B \neq \emptyset\), nominal typing
is the capability-maximizing choice.} Protocol is dominated. Duck typing
is incoherent. The decision follows from the capability analysis, not
from project size or aesthetic preference.

\subsection{Case Study: TypeScript's Design
Tension}\label{case-study-typescripts-design-tension}

TypeScript presents a puzzle: it has explicit inheritance
(\texttt{class\ B\ extends\ A}) but uses structural subtyping. Is this a
valid design tradeoff, or an architectural tension with measurable
consequences?
The runtime model (JavaScript prototypes) preserves \(B\) and nominal
identity (via \texttt{instanceof}), while the static checker erases \(B\)
when computing compatibility
\cite{tsHandbookTypeCompatibility,bierman2014typescript}. Per
Definition 8.3 this is incoherence.

\textbf{Definition 8.3 (Type System Coherence).} A type system is
\emph{coherent} with respect to a language construct if the type
system's judgments align with the construct's runtime semantics.
Formally: if construct \(C\) creates a runtime distinction between
entities \(A\) and \(B\), a coherent type system also distinguishes
\(A\) and \(B\).

\textbf{Definition 8.4 (Type System Tension).} A type system exhibits
\emph{tension} when it is incoherent (per Definition 8.3) AND users
create workarounds to restore the missing distinctions.

\paragraph{8.7.1 The Tension Analysis}\label{the-tension-analysis}

TypeScript's design exhibits three measurable tensions:

\textbf{Tension 1: Incoherence per Definition 8.3.}

\begin{verbatim}
class A \{ x: number = 1; \}
class B \{ x: number = 1; \}

// Runtime: instanceof creates distinction
const b = new B();
console.log(b instanceof A);  // false {- different classes}

// Type system: no distinction
function f(a: A) \{ \}
f(new B());  // OK {- same structure}
\end{verbatim}

The \texttt{class} keyword creates a runtime distinction
(\texttt{instanceof} returns \texttt{false}). The type system does not
reflect this distinction. Per Definition 8.3, this is incoherence: the
construct (\texttt{class}) creates a runtime distinction that the type
system ignores.

\textbf{Tension 2: Workaround existence per Definition 8.4.}

TypeScript programmers use ``branding'' to restore nominal distinctions:

\begin{verbatim}
// Workaround: add a private field to force nominal distinction
class StepWellFilterConfig extends WellFilterConfig \{
    private \_\_brand!: void;  // Forces nominal identity
\}

// Now TypeScript treats them as distinct (private field differs)
\end{verbatim}

The existence of this workaround demonstrates Definition 8.4: users
create patterns to restore distinctions the type system fails to
provide. TypeScript GitHub issue \#202 (2014) and PR \#33038 (2019)
request or experiment with native nominal types
\cite{tsIssue202,tsPR33038}, confirming the workaround is widespread.

\textbf{Tension 3: Measurable consequence.}

The \texttt{extends} keyword is provided but ignored by the type
checker. This is information-theoretically suboptimal per our framework:
the programmer declares a distinction (\texttt{extends}), the type
system discards it, then the programmer re-introduces a synthetic
distinction (\texttt{\_\_brand}). The same information is encoded twice
with different mechanisms.

\paragraph{8.7.2 Formal Characterization}\label{formal-characterization}

\textbf{Theorem 8.7 (TypeScript Incoherence).} TypeScript's class-based
type system is incoherent per Definition 8.3.

\emph{Proof.} 1. TypeScript's \texttt{class\ A} creates a runtime entity
with nominal identity (JavaScript prototype) 2. \texttt{instanceof\ A}
checks this nominal identity at runtime 3. TypeScript's type system uses
structural compatibility for class types 4. Therefore: runtime
distinguishes \texttt{A} from structurally-identical \texttt{B}; type
system does not 5. Per Definition 8.3, this is incoherence.
\(\blacksquare\)

\textbf{Corollary 8.7.1 (Branding Validates Tension).} The prevalence of
branding patterns in TypeScript codebases empirically validates the
tension per Definition 8.4.

\emph{Evidence.} TypeScript GitHub issue \#202 (2014, 1,200+ reactions)
and PR \#33038 (2019) request native nominal types
\cite{tsIssue202,tsPR33038}. The \texttt{@types} ecosystem includes
branded type utilities (\texttt{ts-brand}, \texttt{io-ts}). This is
observed community behavior consistent with the predicted tension.

\paragraph{8.7.3 Implications for Language
Design}\label{implications-for-language-design-1}

TypeScript's tension is an intentional design decision for JavaScript
interoperability. The structural type system allows gradual adoption in
untyped JavaScript codebases. However, TypeScript has \texttt{class}
with \texttt{extends}---meaning \(B \neq \emptyset\). Our theorems
apply: nominal typing strictly dominates (Theorem 3.5).

The tension manifests in practice: programmers use \texttt{class}
expecting nominal semantics, receive structural semantics, then add
branding to restore nominal behavior. Our theorems predict this: Theorem
3.4 shows that when \texttt{bases} exist, nominal typing strictly
dominates structural typing; TypeScript violates this optimality,
causing measurable friction. The branding idiom is programmers manually
recovering capabilities the language architecture foreclosed.

\textbf{The lesson:} Languages adding \texttt{class} syntax should
consider whether their type system will be coherent (per Definition 8.3)
with the runtime semantics of class identity. Structural typing is
correct for languages without inheritance (Go). For languages with
inheritance, coherence requires nominal typing or explicit documentation
of the intentional tension.

\subsection{Mixins with MRO Strictly Dominate Object
Composition}\label{mixins-with-mro-strictly-dominate-object-composition}

The ``composition over inheritance'' principle from the Gang of Four~\cite{gamma1994design} has become software engineering dogma. We demonstrate this
principle is incorrect for behavior extension in languages with explicit
MRO.

\paragraph{8.8.1 Formal Model: Mixin vs
Composition}\label{formal-model-mixin-vs-composition}

\textbf{Definition 8.1 (Mixin).} A mixin is a class designed to provide
behavior via inheritance, with no standalone instantiation. Mixins are
composed via the bases axis, resolved deterministically via MRO.

\begin{verbatim}
\# Mixin: behavior provider via inheritance
class LoggingMixin:
    def process(self):
        print(f"Logging: \{self\}")
        super().process()

class CachingMixin:
    def process(self):
        if cached := self.\_check\_cache():
            return cached
        result = super().process()
        self.\_cache(result)
        return result

\# Composition via bases (single decision point)
class Handler(LoggingMixin, CachingMixin, BaseHandler):
    pass  \# MRO: Handler $\backslash{rightarrow$ Logging $\backslash{}rightarrow$ Caching $\backslash{}rightarrow$ Base}
\end{verbatim}

\textbf{Definition 8.2 (Object Composition).} Object composition
delegates to contained objects, with manual call-site dispatch for each
behavior.

\begin{verbatim}
\# Composition: behavior provider via delegation
class Handler:
    def \_\_init\_\_(self):
        self.logger = Logger()
        self.cache = Cache()

    def process(self):
        self.logger.log(self)  \# Manual dispatch point 1
        if cached := self.cache.check():  \# Manual dispatch point 2
            return cached
        result = self.\_do\_process()
        self.cache.store(key, result)  \# Manual dispatch point 3
        return result
\end{verbatim}

\paragraph{8.8.2 Capability Analysis}\label{capability-analysis}

\textbf{What composition provides:} 1. {[}PASS{]} Behavior extension
(via delegation) 2. {[}PASS{]} Multiple behaviors combined

\textbf{What mixins provide:} 1. {[}PASS{]} Behavior extension (via
super() linearization) 2. {[}PASS{]} Multiple behaviors combined 3.
{[}PASS{]} \textbf{Deterministic conflict resolution} (C3 MRO) ---
\textbf{composition cannot provide} 4. {[}PASS{]} \textbf{Single
decision point} (class definition) --- \textbf{composition has n call
sites} 5. {[}PASS{]} \textbf{Provenance via MRO} (which mixin provided
this behavior?) --- \textbf{composition cannot provide} 6. {[}PASS{]}
\textbf{Exhaustive enumeration} (list all mixed-in behaviors via
\texttt{\_\_mro\_\_}) --- \textbf{composition cannot provide}

\textbf{Addressing runtime swapping:} A common objection is that
composition allows ``swapping implementations at runtime''
(\texttt{handler.cache\ =\ NewCache()}). This is orthogonal to the
dominance claim for two reasons:

\begin{enumerate}
\def\labelenumi{\arabic{enumi}.}
\item
  \textbf{Mixins can also swap at runtime} via class mutation:
  \texttt{Handler.\_\_bases\_\_\ =\ (NewLoggingMixin,\ CachingMixin,\ BaseHandler)}
  or via \texttt{type()} to create a new class dynamically. Python's
  class system is mutable.
\item
  \textbf{Runtime swapping is a separate axis.} The dominance claim
  concerns \emph{static behavior extension}---adding logging, caching,
  validation to a class. Whether to also support runtime reconfiguration
  is an orthogonal requirement. Systems requiring runtime swapping can
  use mixins for static extension AND composition for swappable
  components. The two patterns are not mutually exclusive.
\end{enumerate}

Therefore: \textbf{Mixin capabilities \(\supset\) Composition
capabilities} (strict superset) for static behavior extension.

\textbf{Theorem 8.1 (Mixin Dominance).} For static behavior extension in
languages with deterministic MRO, mixin composition strictly dominates
object composition.

\emph{Proof.} Let \(\mathcal{M}\) = capabilities of mixin composition
(inheritance + MRO). Let \(\mathcal{C}\) = capabilities of object
composition (delegation).

Mixins provide: 1. Behavior extension (same as composition) 2.
Deterministic conflict resolution via MRO (composition cannot provide)
3. Provenance via MRO position (composition cannot provide) 4. Single
decision point for ordering (composition has \(n\) decision points) 5.
Exhaustive enumeration via \texttt{\_\_mro\_\_} (composition cannot
provide)

Therefore \(\mathcal{C} \subset \mathcal{M}\) (strict subset). By the
same argument as Theorem 3.5 (Strict Dominance), choosing composition
forecloses capabilities for zero benefit. \(\blacksquare\)

\textbf{Corollary 8.1.1 (Runtime Swapping Is Orthogonal).} Runtime
implementation swapping is achievable under both patterns: via object
attribute assignment (composition) or via class mutation/dynamic type
creation (mixins). Neither pattern forecloses this capability.

\paragraph{8.8.3 Connection to Typing
Discipline}\label{connection-to-typing-discipline}

\textbf{The parallel to Theorem 3.5 is exact:}

\small
\begin{longtable}[]{@{}
  >{\raggedright\arraybackslash}p{(\linewidth - 2\tabcolsep) * \real{0.4634}}
  >{\raggedright\arraybackslash}p{(\linewidth - 2\tabcolsep) * \real{0.5366}}@{}}
\toprule\noalign{}
\begin{minipage}[b]{\linewidth}\raggedright
Typing Disciplines
\end{minipage} & \begin{minipage}[b]{\linewidth}\raggedright
Architectural Patterns
\end{minipage} \\
\midrule\noalign{}
\endhead
\bottomrule\noalign{}
\endlastfoot
Structural typing checks only namespace (shape) & Composition checks
only namespace (contained objects) \\
Nominal typing checks namespace + bases (MRO) & Mixins check namespace +
bases (MRO) \\
Structural cannot provide provenance & Composition cannot provide
provenance \\
Nominal strictly dominates & Mixins strictly dominate \\
\end{longtable}

\textbf{Theorem 8.2 (Unified Dominance Principle).} In class systems
with explicit inheritance (bases axis), mechanisms using bases strictly
dominate mechanisms using only namespace.

\emph{Proof.} Let \(B\) = bases axis, \(S\) = namespace axis. Let
\(D_S\) = discipline using only \(S\) (structural typing or
composition). Let \(D_B\) = discipline using \(B + S\) (nominal typing
or mixins).

\(D_S\) can only distinguish types/behaviors by namespace content.
\(D_B\) can distinguish by namespace content AND position in inheritance
hierarchy.

Therefore \(\text{capabilities}(D_S) \subset \text{capabilities}(D_B)\)
(strict subset). \(\blacksquare\)

\subsection{Validation: Alignment with Python's Design
Philosophy}\label{validation-alignment-with-pythons-design-philosophy}

Our formal results align with Python's informal design philosophy,
codified in PEP 20 (``The Zen of Python''). This alignment validates
that the abstract model captures real constraints.

\textbf{``Explicit is better than implicit''} (Zen line 2). ABCs require
explicit inheritance declarations (\texttt{class\ Config(ConfigBase)}),
making type relationships visible in code. Duck typing relies on
implicit runtime checks
(\texttt{hasattr(obj,\ \textquotesingle{}validate\textquotesingle{})}),
hiding conformance assumptions. Our Theorem 3.5 formalizes this:
explicit nominal typing provides capabilities that implicit shape-based
typing cannot.

\textbf{``In the face of ambiguity, refuse the temptation to guess''}
(Zen line 12). Duck typing \emph{guesses} interface conformance via
runtime attribute probing. Nominal typing refuses to guess, requiring
declared conformance. Our provenance impossibility result (Corollary
6.3) proves that guessing cannot distinguish structurally identical
types with different inheritance.

\textbf{``Errors should never pass silently''} (Zen line 10). ABCs
fail-loud at instantiation
(\texttt{TypeError:\ Can\textquotesingle{}t\ instantiate\ abstract\ class\ with\ abstract\ method\ validate}).
Duck typing fails-late at attribute access, possibly deep in the call
stack. Our complexity theorems (Section 4) formalize this: nominal
typing has O(1) error localization, while duck typing has \(\Omega\)(n)
error sites.

\textbf{``There should be one-- and preferably only one --obvious way to
do it''} (Zen line 13). Our decision procedure (Section 2.5.1) provides
exactly one obvious way: when \(B \neq \emptyset\), use nominal typing.

\textbf{Historical validation:} Python's evolution confirms our
theorems. Python 1.0 (1991) had only duck typing---an incoherent
non-discipline (Theorem 2.10d). Python 2.6 (2007) added ABCs because
duck typing was insufficient for large codebases. Python 3.8 (2019)
added Protocols for retrofit scenarios---coherent structural typing to
replace incoherent duck typing. This evolution from incoherent
\(\rightarrow\) nominal \(\rightarrow\) nominal+structural exactly
matches our formal predictions.

\subsection{Connection to Gradual
Typing}\label{connection-to-gradual-typing}

Our results connect to the gradual typing literature (Siek \& Taha 2006,
Wadler \& Findler 2009). Gradual typing addresses adding types to
existing untyped code. Our theorems address which discipline to use when
\(B \neq \emptyset\).

\textbf{The complementary relationship:}

\small
\begin{longtable}[]{@{}
  >{\raggedright\arraybackslash}p{(\linewidth - 4\tabcolsep) * \real{0.2564}}
  >{\raggedright\arraybackslash}p{(\linewidth - 4\tabcolsep) * \real{0.3846}}
  >{\raggedright\arraybackslash}p{(\linewidth - 4\tabcolsep) * \real{0.3590}}@{}}
\toprule\noalign{}
\begin{minipage}[b]{\linewidth}\raggedright
Scenario
\end{minipage} & \begin{minipage}[b]{\linewidth}\raggedright
Gradual Typing
\end{minipage} & \begin{minipage}[b]{\linewidth}\raggedright
Our Theorems
\end{minipage} \\
\midrule\noalign{}
\endhead
\bottomrule\noalign{}
\endlastfoot
Untyped code (\(B = \emptyset\)) & {[}PASS{]} Applicable & {[}N/A{]} No
inheritance \\
Typed code (\(B \neq \emptyset\)) & {[}N/A{]} Already typed & {[}PASS{]}
Nominal dominates \\
\end{longtable}

\textbf{Gradual typing's insight:} When adding types to untyped code,
the dynamic type \texttt{?} allows gradual migration. This applies when
\(B = \emptyset\) (no inheritance structure exists yet).

\textbf{Our insight:} When \(B \neq \emptyset\), nominal typing strictly
dominates. This includes ``retrofit'' scenarios with external
types---adapters make nominal typing available (Theorem 2.10j).

\textbf{The unified view:} Gradual typing and nominal typing address
orthogonal concerns: - Gradual typing: Typed vs untyped
(\(B = \emptyset\) $\rightarrow$ \(B \neq \emptyset\) migration) - Our theorems:
Which discipline when \(B \neq \emptyset\) (answer: nominal)

\textbf{Theorem 8.3 (Gradual-Nominal Complementarity).} Gradual typing
and nominal typing are complementary, not competing. Gradual typing
addresses the presence of types; our theorems address which types to
use.

\emph{Proof.} Gradual typing's dynamic type \texttt{?} allows structural
compatibility with untyped code where \(B = \emptyset\). Once
\(B \neq \emptyset\) (inheritance exists), our theorems apply: nominal
typing strictly dominates (Theorem 3.5), and adapters eliminate the
retrofit exception (Theorem 2.10j). The two address different questions.
\(\blacksquare\)

\begin{center}\rule{0.5\linewidth}{0.5pt}\end{center}

\subsection{Connection to Leverage Framework}\label{sec:leverage-connection}

The strict dominance of nominal typing (Theorem 2.10j) is an instance of a more
general principle: \emph{leverage maximization}.

Define \textbf{leverage} as $L = |\text{Capabilities}| / \text{DOF}$, where DOF
(Degrees of Freedom) counts independent encoding locations for type information.
Both typing disciplines have similar DOF (both require type declarations at use
sites), but nominal typing provides 4 additional capabilities (provenance,
identity, enumeration, conflict resolution). Therefore:
\[
L(\text{nominal}) = \frac{5}{1} > \frac{1}{1} = L(\text{duck})
\]

The leverage framework (see companion paper) proves that for any architectural
decision, the optimal choice maximizes leverage. This paper proves the
\emph{instance}; the companion paper proves the \emph{metatheorem} that leverage
maximization is universally optimal.

\textbf{Theorem 8.4 (Typing as Leverage Instance).} The strict dominance of
nominal typing (Theorem 2.10j) is an instance of the Leverage Maximization
Principle.

\emph{Proof.} By Theorem 2.10j, nominal typing provides a strict superset of
capabilities at equivalent cost. This is exactly the condition for higher
leverage: $L(\text{nominal}) > L(\text{duck})$. By the Leverage Maximization
Principle, nominal typing is therefore optimal. $\blacksquare$
