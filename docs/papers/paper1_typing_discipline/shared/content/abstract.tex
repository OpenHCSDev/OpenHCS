\begin{abstract}
\textbf{Theorem.} The type system $(B, S)$ (Bases and Namespace) is the unique minimal complete representation of class-based object semantics.

\textbf{Proof structure:}
\begin{enumerate}
\item \emph{Completeness}: $(B, S)$ answers all typing queries. Type names add no information; they are computable from $(B, S)$.
\item \emph{Minimality}: Neither $B$ nor $S$ alone suffices. Provenance requires $B$. Membership requires $S$. Removing either axis makes some query unanswerable.
\item \emph{Uniqueness}: Any complete system contains $(B, S)$ or is isomorphic to it. There is no alternative.
\end{enumerate}

\textbf{Principal results (machine-checked, 0 sorries):}

\begin{itemize}
\item \textbf{Theorem 3.13 (Provenance Impossibility):} No system without $B$ can compute provenance. This is information-theoretic: the input lacks the data.
\item \textbf{Theorem 3.19 (Capability Partition):} The set of queries partitions exactly into $S$-sufficient and $B$-required. Tertium non datur.
\item \textbf{Theorem 3.24 (Error Localization Lower Bound):} Duck typing requires $\Omega(n)$ inspections to localize errors. Nominal typing achieves $O(1)$. The gap is unbounded.
\item \textbf{Theorem (Minimality $\Rightarrow$ Orthogonality):} Every minimal complete axis set is orthogonal. Non-orthogonal systems contain redundancy and are therefore not minimal.
\end{itemize}

\textbf{Novel Axis $H$ (Hierarchy):} For systems with containment trees, we prove:
\begin{itemize}
\item \textbf{Theorem 3.61 (H Necessity):} There exist queries answerable with $H$ that are impossible without $H$. This is information-theoretic: $(B, S)$ lacks the data.
\item \textbf{Theorem 3.62 (H Orthogonality):} $H$ is not derivable from $B$ or $S$. No lattice homomorphism exists. $H$ is a genuinely new axis.
\item \textbf{Theorem 3.63 (Uniqueness):} $(B, S, H)$ is the unique minimal complete system for hierarchical configuration. There is no alternative.
\end{itemize}
\textbf{Central Result (Axis Derivation):} Axes are not designed. They are \emph{derived} from domain requirements. $B$ emerges when the domain requires provenance. $S$ emerges when the domain requires membership. $H$ emerges when the domain requires hierarchical visibility. The framework computes the minimal complete axis set for any domain.

\textbf{Implications:}
\begin{enumerate}
\item \textbf{Strict dominance.} Unused axes have zero cost. Nominal typing includes $B$. If provenance is needed, $B$ is required. If not needed, $B$ costs nothing. Nominal strictly dominates structural unconditionally.
\item \textbf{Duck typing.} Duck typing is the empty axis set $A = \emptyset$. It answers zero typing queries. Error localization is $\Omega(n)$; nominal achieves $O(1)$. The gap is unbounded.
\item \textbf{Uniqueness.} For any domain $D$, the minimal complete axis set $A_D$ is unique and computable from $D$.
\item \textbf{Fixed axis sets.} A type system with fixed axis set $A$ is incomplete for domains requiring axes outside $A$. Incompleteness is certain for some domain.
\item \textbf{Parametric completeness.} A type system is complete for all domains iff parameterized: $\forall A.\, \mathsf{TypeSystem}(A)$. The API is uniform across axis sets. Query answering is $O(k)$ for $k$ axes. Orthogonality guarantees no axis interaction in query evaluation.
\item \textbf{Axis derivation.} $D \mapsto A_D$ is deterministic. Axes are computed, not chosen.
\end{enumerate}

\textbf{Corollary (Forced Solution):} For any domain $D$, Theorem~3.63 establishes existential uniqueness: $\exists! A$ such that $\text{minimal}(A, D)$. This makes the typing discipline mathematically determined, not designed. Given completeness and minimality as requirements, the solution is forced by the domain structure. Claiming ``typing discipline is a matter of preference'' while accepting the uniqueness theorem instantiates $P \land \neg P$: uniqueness entails $\neg\exists$ alternatives; preference presupposes $\exists$ alternatives. The mathematics admits no choice.

All proofs in Lean 4 (2700+ lines, 142+ theorems, 0 \texttt{sorry}).

\textbf{Keywords:} type systems, nominal typing, structural typing, matroid theory, impossibility theorems, formal verification
\end{abstract}




