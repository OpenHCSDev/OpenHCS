\begin{abstract}
We extend classical rate-distortion theory to a discrete classification setting with three resources: \emph{tag rate} $L$ (bits of storage per entity), \emph{identification cost} $W$ (queries to determine class membership), and \emph{distortion} $D$ (misidentification probability). The fundamental question is: how do these resources trade off when an observer must identify an entity's class from limited observations?

\textbf{Information barrier (zero-error identifiability).} When distinct classes share identical attribute profiles, no algorithm, regardless of computational power, can identify the class from attribute queries alone. Formally: if $\pi$ is not injective on classes, then zero-error identification from attribute queries alone is impossible.

\textbf{Rate-identification tradeoff.} We identify the unique Pareto-optimal zero-error point in the $(L, W, D)$ tradeoff space and describe the induced tradeoff geometry. A nominal tag of $L = \lceil \log_2 k \rceil$ bits for $k$ classes yields $W = O(1)$ with $D = 0$. Without tags ($L = 0$), zero-error identification requires $W = \Omega(d)$ attribute queries, where $d$ is the distinguishing dimension; in the worst case $d = n$ (the ambient attribute count), giving $W = \Omega(n)$. In the presence of attribute collisions, any tag-free scheme incurs $D > 0$.

\textbf{Converse.} In any information-barrier domain, any scheme achieving $D = 0$ requires $L \geq \log_2 k$ bits. This is tight: nominal tagging achieves the bound with $W = O(1)$.

\textbf{Matroid structure.} Minimal sufficient query sets form the bases of a matroid. The \emph{distinguishing dimension} (the common cardinality of all minimal query sets) is well-defined, connecting to zero-error source coding via graph entropy.

\textbf{Applications.} The theory instantiates to type systems (duck vs.\ nominal typing), databases (attribute vs.\ key lookup), and biological taxonomy (phenotype vs.\ species identifier). The unbounded gap $\Omega(d)$ vs.\ $O(1)$ (with a worst-case family where $d = n$) explains convergence toward hybrid systems combining structural observation with nominal tagging.

All results are machine-checked in Lean 4 (6,000+ lines, 0 \texttt{sorry}).

\textbf{Keywords:} rate-distortion theory, identification capacity, zero-error source coding, query complexity, matroid structure, classification systems
\end{abstract}
