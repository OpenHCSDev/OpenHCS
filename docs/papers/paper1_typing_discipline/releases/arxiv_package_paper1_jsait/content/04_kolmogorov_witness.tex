\subsection{Witness Cost for Type Identity}

Recall from Section 2 that the witness cost $W(P)$ is the minimum number of primitive queries required to compute property $P$. For type identity, we ask: what is the minimum number of queries to determine if two values have the same type?

\begin{theorem}[Nominal-Tag Observers Achieve Minimum Witness Cost]
Nominal-tag observers achieve the minimum witness cost for type identity:
$$W_{\text{eq}} = O(1)$$

Specifically, the witness is a single tag read: compare $\text{tag}(v_1) = \text{tag}(v_2)$.

Interface-only observers require $W_{\text{eq}} = \Omega(d)$ where $d$ is the distinguishing dimension (and $d \le n$, with worst-case $d = n$).
\end{theorem}

\begin{proof}
See Lean formalization: \texttt{proofs/nominal\_resolution.lean}. The proof shows:
\begin{enumerate}
\item Nominal-tag access is a single primitive query
\item Interface-only observers must query at least $d$ interfaces in the worst case (a generic strategy queries all $n$)
\item No shorter witness exists for interface-only observers (by the information barrier)
\end{enumerate}
\end{proof}

\subsection{Witness Cost Comparison}

\begin{table}[h]
\centering
\begin{tabular}{|l|c|c|}
\hline
\textbf{Observer Class} & \textbf{Witness Procedure} & \textbf{Witness Cost $W$} \\
\hline
Nominal-tag & Single tag read & $O(1)$ \\
Interface-only & Query a distinguishing set & $\Omega(d)$ \\
\hline
\end{tabular}
\caption{Witness cost for type identity by observer class.}
\end{table}

The Lean 4 formalization (Appendix~\ref{sec:lean}) provides a machine-checked proof that nominal-tag access minimizes witness cost for type identity.
