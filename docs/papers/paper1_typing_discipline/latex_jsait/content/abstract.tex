\begin{abstract}
Classification systems---type systems, database schemas, biological taxonomies, knowledge graphs---must answer queries about entities using a fixed set of observable attributes. We prove fundamental limits on what such systems can compute.

\textbf{Impossibility.} An observer limited to attribute-membership queries cannot determine entity identity when distinct entities share identical attribute profiles. This is an information barrier, not a computational limitation: no algorithm can extract information the observations do not contain.

\textbf{Optimality.} A single additional primitive---a nominal tag identifying each entity's class---reduces witness cost from $\Omega(n)$ to $O(1)$. We prove this is Pareto-optimal in the $(L, W, D)$ tradeoff space (tag length, witness cost, semantic distortion).

\textbf{Structure.} Minimal complete observation sets form the bases of a matroid. All such sets have equal cardinality; the ``semantic dimension'' of a classification problem is well-defined.

\textbf{Universality.} The results apply to any classification system: programming language runtimes (type identity), databases (primary keys), biological taxonomy (species identity), library classification (ISBN). Type systems are one instantiation; the theorems are general.

All results are machine-checked in Lean 4 (6,000+ lines, 0 \texttt{sorry}).

\textbf{Keywords:} classification theory, information barriers, witness complexity, matroid structure, formal verification
\end{abstract}

