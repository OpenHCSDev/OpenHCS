\begin{abstract}
We study zero-error class identification under constrained observations with three resources: tag rate $L$ (bits per entity), identification cost $W$ (attribute queries), and distortion $D$ (misidentification probability). We prove an information barrier: if the attribute-profile map $\pi$ is not injective on classes, then attribute-only observation cannot identify class identity with zero error. Let $A_\pi := \max_u |\{c : \pi(c)=u\}|$ be collision multiplicity. Any $D=0$ scheme must satisfy $L \ge \log_2 A_\pi$, and this bound is tight. In maximal-barrier domains ($A_\pi=k$), the nominal point $(L,W,D)=(\lceil\log_2 k\rceil,O(1),0)$ is the unique Pareto-optimal zero-error point. Without tags ($L=0$), zero-error identification requires $W=\Omega(d)$ queries, where $d$ is the distinguishing dimension (worst case $d=n$, so $W=\Omega(n)$). Minimal sufficient query sets form the bases of a matroid, making $d$ well-defined and linking the model to zero-error source coding via graph entropy. We also state fixed-axis incompleteness: a fixed observation axis is complete only for axis-measurable properties. Results instantiate to databases, biology, typed software systems, and model registries, and are machine-checked in Lean 4 (6589 lines, 296 theorem/lemma statements, 0 \texttt{sorry}).

\textbf{Keywords:} rate-distortion theory, identification capacity, zero-error source coding, query complexity, matroid structure, classification systems
\end{abstract}
