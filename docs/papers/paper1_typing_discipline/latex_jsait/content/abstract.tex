\begin{abstract}
Programming languages implement implicit compression schemes for semantic information. A \emph{type system} compresses the space of possible program behaviors into equivalence classes, enabling static reasoning. We prove that these compression schemes exhibit matroid structure: minimal complete axis sets have equal cardinality.

The key result is a Kolmogorov-optimal witness theorem: the Python \texttt{type()} operation achieves minimum description length (1 AST node) for type identity queries. Alternative compression schemes (structural typing, duck typing) require asymptotically larger witnesses.

We formalize a rate-distortion framework where:
\begin{itemize}
\item \textbf{Rate} = bits required to specify type identity
\item \textbf{Distortion} = probability of semantic misclassification
\end{itemize}

Nominal typing (compression via identity) achieves optimal rate-distortion: zero distortion at minimal rate. Structural typing requires unbounded rate to achieve zero distortion.

All results are machine-checked in Lean 4 (\textasciitilde3,000 lines). This is the first information-theoretic analysis of programming language type systems.

\textbf{Keywords:} Kolmogorov complexity, semantic compression, matroid theory, rate-distortion, type systems, Lean 4
\end{abstract}

