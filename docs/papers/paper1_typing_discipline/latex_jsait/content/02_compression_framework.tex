\subsection{Formal Model: Observations and Equivalence}

Let $\mathcal{V}$ denote the space of all program values. An \emph{observation} is a predicate $\phi: \mathcal{V} \to \{0,1\}$ that tests interface membership.

\begin{definition}[Interface-only equivalence]
Values $x, y \in \mathcal{V}$ are interface-equivalent, written $x \sim y$, iff $\phi(x) = \phi(y)$ for all interface observations $\phi$.
\end{definition}

An interface-only procedure can only distinguish values that are not interface-equivalent. Therefore, any property computed by an interface-only procedure must be constant on $\sim$-equivalence classes.

\subsection{Witness Description Length}

A \emph{witness} for a property $P$ is a program (represented as an AST) that, given access to a value, computes $P$.

\begin{definition}[Witness description length]
The witness description length of property $P$ is $W(P) = \min \{ |w| : w \text{ is a witness program for } P \}$, where $|w|$ is the AST size under a fixed encoding.
\end{definition}

\begin{remark}[Connection to algorithmic information theory]
Witness description length is related to Kolmogorov complexity, but differs in that the encoding and reference machine are fixed (not universal). This makes $W$ a concrete, computable quantity suitable for comparing practical systems.
\end{remark}

\subsection{Rate–Witness–Distortion Tradeoff}

We analyze type identity checking under three dimensions:

\begin{definition}[Tag length]
The tag length $L$ is the number of machine words required to store a type identifier per value. (Under a fixed word size $w$, this corresponds to $\Theta(w)$ bits.)
\end{definition}

\begin{definition}[Witness length]
The witness length $W$ is the minimum AST size to implement type identity checking (Definition above).
\end{definition}

\begin{definition}[Distortion]
The distortion $D$ is a worst-case semantic failure flag:
\[
D = 0 \iff \forall v_1, v_2 \,[\, \text{type}(v_1) = \text{type}(v_2) \Rightarrow \text{behavior}(v_1) \equiv \text{behavior}(v_2) \,]
\]
Otherwise $D = 1$. Here $\text{behavior}(v)$ denotes the observable behavior of $v$ under program execution (e.g., method dispatch outcomes).
\end{definition}

A type system is characterized by a point $(L, W, D)$ in this three-dimensional space. The question is: which points are achievable, and which are Pareto-optimal?

