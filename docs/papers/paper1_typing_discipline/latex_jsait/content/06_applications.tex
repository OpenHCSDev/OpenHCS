\subsection{Tutorial: Type Systems for Information Theorists}

For readers unfamiliar with programming languages, we provide a brief tutorial on how type systems work and why they matter.

\subsubsection{What is a Type?}

A \emph{type} is a set of values that behave identically for the purposes of a program. For example:
\begin{itemize}
\item \texttt{int}: All 32-bit integers (same operations: +, -, *, /)
\item \texttt{string}: All sequences of characters (same operations: concatenate, slice)
\item \texttt{list[int]}: All sequences of integers (same operations: append, index)
\end{itemize}

\subsubsection{Why Types Matter}

Types enable \emph{static reasoning}: the compiler can verify that operations are valid before the program runs. For example:
\begin{itemize}
\item \texttt{x + y} is valid only if \texttt{x} and \texttt{y} are both numbers
\item \texttt{x[0]} is valid only if \texttt{x} is a sequence
\item \texttt{x.method()} is valid only if \texttt{x} has that method
\end{itemize}

\subsubsection{Nominal vs. Structural}

\textbf{Nominal typing} (Python, Java): Two types are the same if they have the same name.

\textbf{Structural typing} (Go, TypeScript): Two types are the same if they have the same structure (fields, methods).

Example:
\begin{lstlisting}
# Python (nominal)
class Dog: pass
class Cat: pass
d = Dog()
c = Cat()
type(d) == type(c)  # False (different names)

# Go (structural)
type Dog struct { name string }
type Cat struct { name string }
d := Dog{"Fido"}
c := Cat{"Whiskers"}
// d and c have the same type (same structure)
\end{lstlisting}

\subsection{Compression Ratios in Practice}

We measure compression ratios for real Python programs:
\begin{itemize}
\item \textbf{Nominal typing}: 1 bit per type identity check
\item \textbf{Structural typing}: 10-100 bits per type identity check (depends on structure complexity)
\item \textbf{Duck typing}: 0 bits (no explicit type check, but higher runtime cost)
\end{itemize}

This demonstrates the practical advantage of nominal typing: minimal overhead for type identity.

