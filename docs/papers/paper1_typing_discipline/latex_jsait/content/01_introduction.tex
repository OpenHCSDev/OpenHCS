\subsection{Observational Constraints and Semantic Inference}

Consider the following inference problem: a procedure observes a program value and must determine its semantic properties (e.g., type identity, provenance). The procedure's access is restricted to \emph{interface-only evidence}: it may query whether the value belongs to certain interface classes, but cannot inspect internal structure or identity tags.

\begin{definition}[Interface-only procedure]
An interface-only procedure is any algorithm whose interaction with a value is limited to interface membership queries.
\end{definition}

\begin{definition}[Indistinguishability]
For values $x, y$, write $x \sim y$ iff every interface membership query returns the same answer on $x$ and on $y$.
\end{definition}

The central question is: \textbf{what semantic properties can an interface-only procedure compute?}

\subsection{The Impossibility Barrier}

\begin{theorem}[Impossibility from interface-only evidence]
Every interface-only procedure is constant on $\sim$-equivalence classes. Consequently, no interface-only procedure can compute any property that differs for some $x \sim y$.
\end{theorem}

This is an information barrier: the restriction is not computational (unbounded time/memory does not help) but informational (the evidence itself is insufficient).

\subsection{The Positive Result: Nominal Tagging}

In contrast, nominal tagging—storing an explicit type identifier per value—provides constant-size evidence for type identity.

\begin{definition}[Witness description length]
Let $W(P)$ denote the minimum description length (e.g., AST size under a fixed encoding) of a witness program for property $P$.
\end{definition}

\begin{theorem}[Constant witness for nominal type identity]
Nominal-tag access admits a constant-length witness for type identity: $W(\text{type-identity}) = O(1)$.
\end{theorem}

\subsection{Main Contributions}

\begin{enumerate}
\item \textbf{Impossibility Theorem}: No interface-only procedure can compute properties that vary within indistinguishability classes (Theorem 1).

\item \textbf{Constant-Witness Result}: Nominal tagging achieves $W(\text{type-identity}) = O(1)$ (Theorem 2).

\item \textbf{Matroid Structure}: The space of semantic queries decomposes as a matroid; all minimal complete query sets have equal cardinality.

\item \textbf{Rate–Witness–Distortion Optimality}: Nominal tagging achieves the unique Pareto-optimal point in the (tag-length, witness-length, error-rate) tradeoff.

\item \textbf{Machine-Checked Proofs}: All results formalized in Lean 4 (\textasciitilde6,000 lines, 265 theorems, 0 sorry).
\end{enumerate}

\subsection{Audience and Scope}

This paper is written for the information theory and compression community. We assume familiarity with matroid theory and basic information-theoretic concepts. We provide concrete instantiations in widely used programming language runtimes (CPython, Java, TypeScript, Rust) as corollaries to the main theorems.

