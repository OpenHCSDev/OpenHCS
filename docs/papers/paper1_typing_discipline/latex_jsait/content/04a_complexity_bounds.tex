\section{Core Theorems}\label{core-theorems}

\subsection{The Error Localization
Theorem}\label{the-error-localization-theorem}

\textbf{Definition 4.1 (Error Location).} Let E(T) be the number of
source locations that must be inspected to find all potential violations
of a type constraint under discipline T.

\textbf{Theorem 4.1 (nominal-tag Typing Complexity).} E(nominal-tag) = O(1).

\emph{Proof.} Under nominal-tag observation, constraint ``x must be an A'' is
satisfied iff type(x) inherits from A. This property is determined at
class definition time, at exactly one location: the class definition of
type(x). If the class does not list A in its bases (transitively), the
constraint fails. One location. \qed

\textbf{Remark:} In type system terminology, nominal-tag observation corresponds to nominal-tag observation.

\textbf{Theorem 4.2 (Interface-Only Declared Complexity).} E(interface-only (declared)) = O(k) where
k = number of classes.

\emph{Proof.} Under interface-only typing with declared interfaces, constraint ``x must satisfy
interface A'' requires checking that type(x) implements all methods in
signature(A). This check occurs at each class definition. For k classes,
O(k) locations. \qed

\textbf{Remark:} In type system terminology, this is called interface-only (declared) observation.

\textbf{Theorem 4.3 (Interface-Only Incoherent Complexity).} E(interface-only) = $\Omega(n)$
where n = number of call sites.

\emph{Proof.} Under interface-only typing, constraint ``x must have method m'' is
encoded as \texttt{hasattr(x,\ "m")} at each call site. There is no
central declaration. For n call sites, each must be inspected. Lower
bound is $\Omega(n)$. \qed

\textbf{Remark:} This incoherent pattern is traditionally called "interface-only observation."

\textbf{Corollary 4.4 (Strict Dominance).} nominal-tag observation strictly
dominates interface-only: E(nominal-tag) = O(1) \textless{} $\Omega(n)$ =
E(interface-only) for all n \textgreater{} 1.

\textbf{Remark:} In type system terminology, this shows nominal-tag observation dominates interface-only observation.

\subsection{The Information Scattering
Theorem}\label{the-information-scattering-theorem}

\textbf{Definition 4.2 (Constraint Encoding Locations).} Let I(T, c) be
the set of source locations where constraint c is encoded under
discipline T.

\textbf{Theorem 4.5 (Interface-Only Incoherent Scattering).} For interface-only typing,
\textbar I(interface-only, c)\textbar{} = O(n) where n = call sites using
constraint c.

\textbf{Remark:} This describes the scattering problem in "interface-only observation."

\emph{Proof.} Each \texttt{hasattr(x,\ "method")} call independently
encodes the constraint. No shared reference. Constraints scale with call
sites. \qed

\textbf{Theorem 4.6 (nominal-tag Typing Centralizes).} For nominal-tag observation,
\textbar I(nominal-tag, c)\textbar{} = O(1).

\emph{Proof.} Constraint c = ``must inherit from A'' is encoded once: in
the ABC/Protocol definition of A. All \texttt{isinstance(x,\ A)} checks
reference this single definition. \qed

\textbf{Remark:} In type system terminology, nominal-tag observation corresponds to nominal-tag observation.

\textbf{Corollary 4.7 (Maintenance Entropy).} interface-only typing maximizes
maintenance entropy; nominal-tag observation minimizes it.


