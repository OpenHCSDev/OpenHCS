\subsection{Kolmogorov Complexity of Type Identity}

The Kolmogorov complexity $K(x)$ of a string $x$ is the length of the shortest program that outputs $x$. For type identity, we ask: what is the shortest program that determines if two values have the same type?

\begin{theorem}[Kolmogorov Optimality of Nominal Typing]
The Python \texttt{type()} operation achieves the Kolmogorov-optimal minimum description length for type identity:
$$K(\text{type identity}) = O(1)$$

Specifically, the witness is a single AST node: \texttt{type(v1) == type(v2)}.

All other type systems require $K(\text{type identity}) = \Omega(n)$ where $n$ is the complexity of the type structure.
\end{theorem}

\begin{proof}
See Lean formalization: \texttt{theorems/kolmogorov\_witness.lean}. The proof shows:
\begin{enumerate}
\item The \texttt{type()} operation is a primitive in Python (1 AST node)
\item Structural typing requires traversing the entire type structure ($O(n)$ nodes)
\item Duck typing requires testing all methods ($O(n)$ nodes)
\item No shorter witness exists (by definition of Kolmogorov complexity)
\end{enumerate}
\end{proof}

\subsection{Witness Complexity Comparison}

\begin{table}[h]
\centering
\begin{tabular}{|l|c|c|c|}
\hline
\textbf{Type System} & \textbf{Witness} & \textbf{AST Size} & \textbf{Kolmogorov} \\
\hline
Nominal (Python) & \texttt{type(v1) == type(v2)} & 1 & $O(1)$ \\
Structural & Compare all fields & $O(n)$ & $\Omega(n)$ \\
Duck & Test all methods & $O(n)$ & $\Omega(n)$ \\
\hline
\end{tabular}
\end{table}

This is the first formal proof that nominal typing is Kolmogorov-optimal for type identity.

