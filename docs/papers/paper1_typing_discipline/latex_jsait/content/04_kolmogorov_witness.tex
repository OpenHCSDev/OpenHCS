\subsection{Witness Description Length for Type Identity}

Recall from Section 2 that the witness description length $W(P)$ is the minimum AST size of a program that computes property $P$. For type identity, we ask: what is the shortest program that determines if two values have the same type?

\begin{theorem}[Nominal Typing Achieves Minimum Witness Length]
Nominal-tag access achieves the minimum witness description length for type identity:
$$W(\text{type identity}) = O(1)$$

Specifically, the witness is a single AST node: \texttt{type(v1) == type(v2)}.

All other type systems require $W(\text{type identity}) = \Omega(n)$ where $n$ is the complexity of the type structure.
\end{theorem}

\begin{proof}
See Lean formalization: \texttt{theorems/nominal\_resolution.lean}. The proof shows:
\begin{enumerate}
\item The nominal-tag access operation is a primitive (1 AST node)
\item Structural typing requires traversing the entire type structure ($O(n)$ nodes)
\item Duck typing requires testing all methods ($O(n)$ nodes)
\item No shorter witness exists (by definition of witness description length)
\end{enumerate}
\end{proof}

\subsection{Witness Complexity Across Type Systems}

\begin{table}[h]
\centering
\begin{tabular}{|l|c|c|}
\hline
\textbf{Type System} & \textbf{Witness Program} & \textbf{Witness Length} \\
\hline
Nominal & \texttt{type(v1) == type(v2)} & $O(1)$ \\
Structural & Compare all fields & $O(n)$ \\
Duck & Test all methods & $O(n)$ \\
\hline
\end{tabular}
\caption{Witness description length for type identity across type systems.}
\end{table}

This is the first formal proof that nominal-tag access minimizes witness description length for type identity.

