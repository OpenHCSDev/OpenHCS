This paper presents an information-theoretic analysis of programming language type systems. We prove three main results:

\begin{enumerate}
\item \textbf{Impossibility Barrier}: No interface-only procedure can compute properties that vary within indistinguishability classes.

\item \textbf{Constant-Witness Result}: Nominal tagging achieves $W(\text{type-identity}) = O(1)$, the minimum witness description length.

\item \textbf{Pareto Optimality}: Nominal typing is the unique Pareto-optimal point in the $(L, W, D)$ tradeoff: minimal tag length, minimal witness length, zero distortion.
\end{enumerate}

\subsection{Implications}

These results have several implications:

\begin{itemize}
\item \textbf{Nominal typing is provably optimal} for type identity checking, not just a design choice.

\item \textbf{Structural typing is provably suboptimal}: it requires unbounded witness length to achieve the same distortion as nominal typing.

\item \textbf{Duck typing trades tag length for distortion}: it reduces tag length but increases misclassification probability.

\item \textbf{No type system can do better than nominal typing} while remaining complete and zero-distortion.

\item \textbf{The barrier is informational, not computational}: even with unbounded time and memory, interface-only procedures cannot overcome the indistinguishability barrier.
\end{itemize}

\subsection{Future Work}

This work opens several directions:

\begin{enumerate}
\item \textbf{Concept Matroids}: Do other programming language concepts (modules, inheritance, generics) exhibit matroid structure?

\item \textbf{Witness Complexity of Other Properties}: Can we formalize the witness complexity of other semantic properties (e.g., provenance, mutability)?

\item \textbf{Hybrid Systems}: Can we design type systems that achieve better $(L, W, D)$ tradeoffs by combining nominal and structural approaches?

\item \textbf{Runtime Verification}: How do runtime type checks affect the witness complexity analysis?
\end{enumerate}

\subsection{Conclusion}

Type systems are semantic compression schemes under observational constraints. By applying information theory, we can formally analyze their optimality. This work demonstrates that nominal typing is not just a design choice, but the provably optimal compression scheme for type identity.

All proofs are machine-verified in Lean 4, providing absolute certainty in the results.

