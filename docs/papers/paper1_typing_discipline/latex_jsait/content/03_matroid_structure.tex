\subsection{Type Axes}

A \emph{type axis} is a semantic dimension along which types can vary. Examples:
\begin{itemize}
\item \textbf{Identity}: Explicit type name or object ID
\item \textbf{Structure}: Field names and types
\item \textbf{Behavior}: Available methods and their signatures
\item \textbf{Scope}: Where the type is defined (module, package)
\item \textbf{Mutability}: Whether instances can be modified
\end{itemize}

A \emph{complete} axis set distinguishes all semantically distinct types. A \emph{minimal complete} axis set is complete with no proper complete subset.

\subsection{Matroid Structure of Type Axes}

\begin{definition}[Axis bases family]
Let $E$ be the set of all type axes. Let $\mathcal{B} \subseteq 2^E$ be the family of minimal complete axis sets.
\end{definition}

\begin{lemma}[Basis exchange]
For any $B_1, B_2 \in \mathcal{B}$ and any $e \in B_1 \setminus B_2$, there exists $f \in B_2 \setminus B_1$ such that $(B_1 \setminus \{e\}) \cup \{f\} \in \mathcal{B}$.
\end{lemma}

\begin{proof}
See Lean formalization: \texttt{proofs/axis\_framework.lean}, lemma \texttt{basis\_exchange}.
\end{proof}

\begin{theorem}[Matroid bases]
$\mathcal{B}$ is the set of bases of a matroid on ground set $E$.
\end{theorem}

\begin{proof}
By the basis-exchange lemma and the standard characterization of matroid bases.
\end{proof}

\begin{corollary}[Well-defined semantic dimension]
All minimal complete axis sets have equal cardinality. Hence the ``semantic dimension'' of a type system is well-defined.
\end{corollary}

\subsection{Compression Optimality}

\begin{corollary}[Compression Optimality]
All minimal complete type systems achieve the same compression ratio. No type system can be strictly more efficient than another while remaining complete.
\end{corollary}

This means: nominal typing, structural typing, and duck typing all achieve the same compression ratio when minimal. The difference is in \emph{witness complexity}, not compression efficiency.

