\subsection{Type Axes}

A \emph{type axis} is a semantic dimension along which types can vary. Examples:
\begin{itemize}
\item \textbf{Identity}: Explicit type name or object ID
\item \textbf{Structure}: Field names and types
\item \textbf{Behavior}: Available methods and their signatures
\item \textbf{Scope}: Where the type is defined (module, package)
\item \textbf{Mutability}: Whether instances can be modified
\end{itemize}

A \emph{complete type system} must distinguish types along all necessary axes. A \emph{minimal complete type system} uses the fewest axes while remaining complete.

\subsection{Matroid Theorem}

\begin{theorem}[Type Axes Form a Matroid]
The set of type axes forms a matroid $M = (E, \mathcal{I})$ where:
\begin{itemize}
\item $E$ = all possible type axes
\item $\mathcal{I}$ = all minimal complete axis sets
\end{itemize}

All bases (minimal complete axis sets) have equal cardinality.
\end{theorem}

\begin{proof}
See Lean formalization: \texttt{theorems/matroid\_structure.lean}. The proof verifies:
\begin{enumerate}
\item Hereditary property: Any subset of a complete set is independent
\item Exchange property: Any two minimal complete sets have equal cardinality
\item Basis property: All maximal independent sets have equal size
\end{enumerate}
\end{proof}

\subsection{Compression Optimality}

\begin{corollary}[Compression Optimality]
All minimal complete type systems achieve the same compression ratio. No type system can be strictly more efficient than another while remaining complete.
\end{corollary}

This means: nominal typing, structural typing, and duck typing all achieve the same compression ratio when minimal. The difference is in \emph{witness complexity}, not compression efficiency.

