\subsection{Type Axes}

A \emph{type axis} is a semantic dimension along which types can vary. Examples:
\begin{itemize}
\item \textbf{Identity}: Explicit type name or object ID
\item \textbf{Structure}: Field names and types
\item \textbf{Behavior}: Available methods and their signatures
\item \textbf{Scope}: Where the type is defined (module, package)
\item \textbf{Mutability}: Whether instances can be modified
\end{itemize}

A \emph{complete type system} must distinguish types along all necessary axes. A \emph{minimal complete type system} uses the fewest axes while remaining complete.

\subsection{Equicardinality Theorem}

\begin{theorem}[Equicardinality of Minimal Complete Axis Sets]
Let $E$ be the set of all type axes. All minimal complete axis sets have equal cardinality.
\end{theorem}

\begin{proof}
See Lean formalization: \texttt{proofs/axis\_framework.lean}. The proof establishes that minimal complete sets are ``semantically orthogonal'' bases; the lemmas \texttt{semantically\_minimal\_implies\_orthogonal} and \texttt{minimal\_complete\_unique\_orthogonal} yield equicardinality.
\end{proof}

\begin{remark}[Matroid structure]
The family of minimal complete axis sets $\mathcal{B}$ satisfies the basis-exchange property: for any $B_1, B_2 \in \mathcal{B}$ and $x \in B_1 \setminus B_2$, there exists $y \in B_2 \setminus B_1$ such that $(B_1 \setminus \{x\}) \cup \{y\} \in \mathcal{B}$. This makes $\mathcal{B}$ the set of bases of a matroid on $E$. Equicardinality then follows from standard matroid theory.
\end{remark}

\subsection{Compression Optimality}

\begin{corollary}[Compression Optimality]
All minimal complete type systems achieve the same compression ratio. No type system can be strictly more efficient than another while remaining complete.
\end{corollary}

This means: nominal typing, structural typing, and duck typing all achieve the same compression ratio when minimal. The difference is in \emph{witness complexity}, not compression efficiency.

