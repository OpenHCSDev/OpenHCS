\subsection{Rate-Distortion Tradeoff}

Following Cover and Thomas, we analyze the rate-distortion frontier for type systems.

\begin{theorem}[Rate-Distortion Optimality]
Nominal typing achieves the unique Pareto-optimal point in the rate-distortion plane:
\begin{itemize}
\item \textbf{Rate}: $R = O(1)$ bits per type
\item \textbf{Distortion}: $D = 0$ (zero misclassification)
\end{itemize}

Structural typing achieves:
\begin{itemize}
\item \textbf{Rate}: $R = O(n)$ bits per type (unbounded)
\item \textbf{Distortion}: $D = 0$ (zero misclassification)
\end{itemize}

Duck typing achieves:
\begin{itemize}
\item \textbf{Rate}: $R = O(1)$ bits per type
\item \textbf{Distortion}: $D > 0$ (positive misclassification probability)
\end{itemize}
\end{theorem}

\begin{proof}
See Lean formalization: \texttt{theorems/rate\_distortion.lean}. The proof verifies:
\begin{enumerate}
\item Nominal achieves $(R, D) = (O(1), 0)$ via \texttt{type()} primitive
\item Structural requires $O(n)$ bits to encode structure
\item Duck typing cannot guarantee zero distortion (runtime behavior varies)
\item No other scheme achieves $(O(1), 0)$
\end{enumerate}
\end{proof}

\subsection{Pareto Frontier}

The rate-distortion frontier shows:
\begin{itemize}
\item Nominal typing dominates all other schemes
\item Structural typing is suboptimal (higher rate, same distortion)
\item Duck typing trades rate for distortion (lower rate, higher distortion)
\end{itemize}

This is the first formal proof that nominal typing is Pareto-optimal for type systems.

