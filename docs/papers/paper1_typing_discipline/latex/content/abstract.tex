\begin{abstract}
\textbf{The Problem.} Classification systems---type systems, ontologies, taxonomies, schemas---operate over fixed sets of classification axes. We prove this architectural choice has unavoidable consequences: \emph{any fixed-axis system is incomplete for some domain}. This is not a limitation of specific implementations; it is an information-theoretic impossibility.

\textbf{The Core Theorems (machine-checked, 0 sorries):}
\begin{itemize}
\item \textbf{Fixed Axis Incompleteness:} For any axis set $A$ and any axis $a \notin A$, there exists a domain $D$ that $A$ cannot serve. The information required to answer $D$'s queries does not exist in $A$.
\item \textbf{Parameterized Immunity:} For any domain $D$, there exists an axis set $A_D$ that is complete for $D$. This set is computable: $A_D = \bigcup_{q \in D} \text{requires}(q)$.
\item \textbf{The Asymmetry:} Fixed systems guarantee failure for some domain. Parameterized systems guarantee success for all domains. One dominates the other absolutely.
\item \textbf{Uniqueness:} For any domain $D$, all minimal complete axis sets have equal cardinality. ``Dimension'' is well-defined for classification problems.
\item \textbf{Minimality $\Rightarrow$ Orthogonality:} Every minimal complete axis set is orthogonal. Orthogonality is not imposed; it is derived from minimality.
\end{itemize}

\textbf{The Prescriptive Force.} These are not design recommendations. They are mathematical necessities:
\begin{enumerate}
\item Given any domain $D$, the required axes are \emph{computable}, not chosen.
\item Missing axes cause \emph{impossibility}, not difficulty. No implementation overcomes a missing axis.
\item The choice of axis-parameterization is \emph{forced} by the requirement of domain-agnosticism.
\end{enumerate}

\textbf{Application to Type Systems.} We instantiate the framework to programming language type systems, proving:
\begin{itemize}
\item $(B, S)$ (Bases and Namespace) is the unique minimal complete representation of class semantics
\item $(B, S, H)$ extends this for hierarchical configuration systems (adding a Scope axis)
\item Nominal typing strictly dominates structural typing when inheritance exists ($B \neq \emptyset$)
\item Duck typing requires $\Omega(n)$ error localization; nominal achieves $O(1)$
\end{itemize}

\textbf{The Broader Claim.} The impossibility theorems apply to \emph{any} classification system with fixed dimensions---not only type systems. Biological taxonomies, library classification schemes, database schemas, knowledge graphs: all are subject to the same constraints. A fixed set of axes guarantees domains that cannot be served.

\textbf{Corollary (Incoherence of Preference).} Claiming ``classification system design is a matter of preference'' while accepting the uniqueness theorem instantiates $P \land \neg P$. Uniqueness entails $\neg\exists$ alternatives; preference presupposes $\exists$ alternatives. The mathematics admits no choice.

All proofs in Lean 4 (2700+ lines, 142+ theorems, 0 \texttt{sorry}).

\textbf{Keywords:} classification theory, impossibility theorems, matroid theory, type systems, formal verification, epistemology
\end{abstract}




