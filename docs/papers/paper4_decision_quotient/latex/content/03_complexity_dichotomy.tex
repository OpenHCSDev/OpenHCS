\section{Complexity Dichotomy}\label{sec:dichotomy}

The hardness results of Section~\ref{sec:hardness} apply to worst-case instances. This section develops a more nuanced picture: a \emph{dichotomy theorem} showing that problem difficulty depends on the size of the minimal sufficient set.

\begin{theorem}[Complexity Dichotomy]\label{thm:dichotomy}
Let $\mathcal{D} = (A, X_1, \ldots, X_n, U)$ be a decision problem with $|S| = N$ states. Let $k^*$ be the size of the minimal sufficient set.

\begin{enumerate}
\item \textbf{Logarithmic case:} If $k^* = O(\log N)$, then SUFFICIENCY-CHECK is solvable in polynomial time.

\item \textbf{Linear case:} If $k^* = \Omega(n)$, then SUFFICIENCY-CHECK requires time $\Omega(2^{n/c})$ for some constant $c > 0$ (assuming ETH).
\end{enumerate}
\end{theorem}

\begin{proof}
\textbf{Part 1 (Logarithmic case):} If $k^* = O(\log N)$, then the number of distinct projections $|S_{I^*}|$ is at most $2^{k^*} = O(N^c)$ for some constant $c$. We can enumerate all projections and verify sufficiency in polynomial time.

\textbf{Part 2 (Linear case):} We establish this via an explicit reduction chain from the Exponential Time Hypothesis.
\end{proof}

\subsection{The ETH Reduction Chain}\label{sec:eth-chain}

The lower bound in Part 2 of Theorem~\ref{thm:dichotomy} follows from a chain of reductions originating in the Exponential Time Hypothesis. We make this chain explicit.

\begin{definition}[Exponential Time Hypothesis (ETH)]
There exists a constant $\delta > 0$ such that 3-SAT on $n$ variables cannot be solved in time $O(2^{\delta n})$~\cite{impagliazzo2001complexity}.
\end{definition}

The chain proceeds as follows:

\begin{enumerate}
\item \textbf{ETH $\Rightarrow$ 3-SAT requires $2^{\Omega(n)}$:} This is the definition of ETH.

\item \textbf{3-SAT $\leq_p$ TAUTOLOGY:} Given 3-SAT formula $\varphi(x_1, \ldots, x_n)$, define $\psi = \neg\varphi$. Then $\varphi$ is satisfiable iff $\psi$ is not a tautology. This is a linear-time reduction preserving the number of variables.

\item \textbf{TAUTOLOGY requires $2^{\Omega(n)}$ (under ETH):} By the contrapositive of step 2, if TAUTOLOGY could be solved in $o(2^{\delta n})$ time, then 3-SAT could be solved in $o(2^{\delta n})$ time, contradicting ETH.

\item \textbf{TAUTOLOGY $\leq_p$ SUFFICIENCY-CHECK:} This is Theorem~\ref{thm:sufficiency-conp}. Given formula $\varphi(x_1, \ldots, x_n)$, we construct a decision problem where:
\begin{itemize}
\item The empty set $I = \emptyset$ is sufficient iff $\varphi$ is a tautology
\item When $\varphi$ is not a tautology, all $n$ coordinates are relevant
\end{itemize}
The reduction is polynomial-time and preserves the number of coordinates.

\item \textbf{SUFFICIENCY-CHECK requires $2^{\Omega(n)}$ (under ETH):} Combining steps 3 and 4: if SUFFICIENCY-CHECK could be solved in $o(2^{\delta n/c})$ time for some constant $c$, then TAUTOLOGY (and hence 3-SAT) could be solved in subexponential time, contradicting ETH.
\end{enumerate}

\begin{proposition}[Tight Constant]\label{prop:eth-constant}
The reduction in Theorem~\ref{thm:sufficiency-conp} preserves the number of variables exactly: an $n$-variable formula yields an $n$-coordinate decision problem. Therefore, the constant $c$ in the $2^{n/c}$ lower bound equals 1:
\[
\text{SUFFICIENCY-CHECK requires time } \Omega(2^{\delta n}) \text{ under ETH}
\]
where $\delta$ is the ETH constant for 3-SAT.
\end{proposition}

\begin{proof}
The TAUTOLOGY reduction (Theorem~\ref{thm:sufficiency-conp}) constructs:
\begin{itemize}
\item State space $S = \{\text{ref}\} \cup \{0,1\}^n$ with $n+1$ coordinates (one extra for the reference state)
\item Query set $I = \emptyset$
\end{itemize}
When $\varphi$ has $n$ variables, the constructed problem has $n+1$ coordinates. The asymptotic lower bound is $2^{\Omega(n)}$ with the same constant $\delta$ from ETH.
\end{proof}

\subsection{Phase Transition}

\begin{corollary}[Phase Transition]\label{cor:phase-transition}
There exists a threshold $\tau \in (0, 1)$ such that:
\begin{itemize}
\item If $k^*/n < \tau$, SUFFICIENCY-CHECK is ``easy'' (polynomial in $N$)
\item If $k^*/n > \tau$, SUFFICIENCY-CHECK is ``hard'' (exponential in $n$)
\end{itemize}
\end{corollary}

\begin{proof}
The logarithmic case (Part 1 of Theorem~\ref{thm:dichotomy}) gives polynomial time when $k^* = O(\log N) = O(\log 2^n) = O(n)$. More precisely, when $k^* \leq c \log N$ for constant $c$, the algorithm runs in time $O(N^c \cdot \text{poly}(n))$.

The linear case (Part 2) gives exponential time when $k^* = \Omega(n)$.

The threshold $\tau$ is implicitly defined by where the polynomial bound $2^{k^*} = N^{k^*/\log N}$ transitions from polynomial to superpolynomial in $N$. This occurs when $k^*/\log N$ exceeds any constant, i.e., when $k^* = \omega(\log N)$.

For Boolean coordinate spaces ($N = 2^n$), the threshold is $\tau = 0$: any $k^* = \omega(\log n)$ yields superpolynomial complexity, while $k^* = O(\log n)$ is tractable.
\end{proof}

\begin{remark}[Sharpness of Dichotomy]
The dichotomy is asymptotically tight under ETH. There are no ``intermediate'' cases:
\begin{itemize}
\item \textbf{Below threshold:} Polynomial (by enumeration)
\item \textbf{At threshold:} Quasipolynomial ($n^{O(\log n)}$)
\item \textbf{Above threshold:} Exponential (by ETH)
\end{itemize}
The quasipolynomial regime at the threshold is measure-zero; almost all instances are either clearly tractable or clearly intractable.
\end{remark}

This dichotomy explains why some domains admit tractable model selection (few relevant variables) while others require heuristics (many relevant variables). The ETH reduction chain makes precise what ``hard'' means: not merely coNP-complete, but requiring $2^{\Omega(n)}$ time under widely-believed complexity assumptions.


