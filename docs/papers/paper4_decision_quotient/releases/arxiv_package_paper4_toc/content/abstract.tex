We characterize the computational complexity of coordinate sufficiency in decision problems within the formal model. Given action set $A$, state space $S = X_1 \times \cdots \times X_n$, and utility $u: A \times S \to \mathbb{R}$, a coordinate set $I$ is \emph{sufficient} if $s_I = s'_I \implies \Opt(s) = \Opt(s')$.

\textbf{The landscape in the formal model:}
\begin{itemize}
\item \textbf{General case:} SUFFICIENCY-CHECK is \coNP-complete; ANCHOR-SUFFICIENCY is $\SigmaP{2}$-complete.
\item \textbf{Tractable cases:} Polynomial-time for bounded action sets under the explicit-state encoding; separable utilities ($u = f + g$) under any encoding; and tree-structured utility with explicit local factors.
\item \textbf{Encoding-regime separation:} Polynomial-time under the explicit-state encoding (polynomial in $|S|$). Under ETH, there exist succinctly encoded worst-case instances witnessed by a strengthened gadget construction (mechanized in Lean; see Appendix~\ref{app:lean}) with $k^*=n$ for which SUFFICIENCY-CHECK requires $2^{\Omega(n)}$ time.
\end{itemize}

The tractable cases are stated with explicit encoding assumptions (Section~\ref{sec:encoding}). Together, these results answer the question ``when is decision-relevant information identifiable efficiently?'' within the stated regimes.

The primary contribution is theoretical: a formalized reduction framework and a complete characterization of the core decision-relevant problems in the formal model (coNP/\(\Sigma_2^P\) completeness and tractable cases under explicit encoding assumptions). The practical corollaries follow from these theorems.

All results are machine-checked in Lean 4 ($\sim$5,000 lines, 200+ theorems).

\textbf{Keywords:} computational complexity, decision theory, polynomial hierarchy, tractability dichotomy, Lean 4
