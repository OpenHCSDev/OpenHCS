\section{The Simplicity Tax: A Conservation Law for Complexity}\label{sec:simplicity-tax}

The hardness results of Sections~\ref{sec:hardness}--\ref{sec:tractable} establish that identifying decision-relevant dimensions is \coNP-complete. This section develops a quantitative consequence: a conservation law for complexity that governs the total work required when using models of varying expressiveness.

\subsection{Definitions}\label{sec:tax-definitions}

\begin{definition}[Problem and Model]\label{def:problem-tool}
A \emph{problem} $P$ has a set of \emph{required dimensions} $R(P) \subseteq \{1,\ldots,n\}$---the coordinates that affect the optimal action. A \emph{model} (or analysis method) $M$ has a set of \emph{native dimensions} $A(M)$---the coordinates it can represent directly.
\end{definition}

\begin{definition}[Expressive Gap]\label{def:expressive-gap}
The \emph{expressive gap} between model $M$ and problem $P$ is:
\[\text{Gap}(M, P) = R(P) \setminus A(M)\]
The \emph{simplicity tax} is $|\text{Gap}(M, P)|$: the number of dimensions the model cannot represent natively.
\end{definition}

\begin{definition}[Complete vs. Incomplete Models]\label{def:complete-incomplete}
Model $M$ is \emph{complete} for problem $P$ if $R(P) \subseteq A(M)$. Otherwise $M$ is \emph{incomplete} for $P$.
\end{definition}

\subsection{The Conservation Theorem}\label{sec:conservation}

\begin{theorem}[Complexity Conservation]\label{thm:tax-conservation}
For any problem $P$ with required dimensions $R(P)$ and any model $M$:
\[|\text{Gap}(M, P)| + |R(P) \cap A(M)| = |R(P)|\]
The required dimensions are partitioned into ``handled natively'' and ``handled externally.'' The total cannot be reduced---only redistributed.
\end{theorem}

\begin{proof}
Set partition: $R(P) = (R(P) \cap A(M)) \cup (R(P) \setminus A(M))$. The sets are disjoint. Cardinality follows. \qed
\end{proof}

\begin{theorem}[Complete Models Pay No Tax]\label{thm:complete-no-tax}
If $M$ is complete for $P$, then $\text{SimplicityTax}(M, P) = 0$.
\end{theorem}

\begin{theorem}[Incomplete Models Pay Positive Tax]\label{thm:incomplete-positive-tax}
If $M$ is incomplete for $P$, then $\text{SimplicityTax}(M, P) > 0$.
\end{theorem}

\begin{theorem}[Linear Growth]\label{thm:tax-grows}
For $n$ decision sites using an incomplete model:
\[\text{TotalExternalWork}(M, P, n) = n \times \text{SimplicityTax}(M, P)\]
Total external work grows linearly in the number of sites.
\end{theorem}

\begin{theorem}[Dominance]\label{thm:complete-dominates}
For any $n > 0$, a complete model has strictly less total work than an incomplete model:
\[\text{TotalExternalWork}(M_{\text{complete}}, P, n) < \text{TotalExternalWork}(M_{\text{incomplete}}, P, n)\]
\end{theorem}

\begin{proof}
Complete: $0 \times n = 0$. Incomplete: $k \times n$ for $k \geq 1$. For $n > 0$: $0 < kn$. \qed
\end{proof}

\subsection{Central vs. Distributed Costs}\label{sec:central-distributed}

When comparing models, two cost components must be distinguished:
\begin{itemize}
\item $H_{\text{central}}$: The one-time cost of constructing or learning the model.
\item $H_{\text{distributed}}$: The per-site cost of handling dimensions not covered by the model.
\end{itemize}

A model with higher $H_{\text{central}}$ but zero $H_{\text{distributed}}$ may have lower total cost than a model with lower $H_{\text{central}}$ but positive $H_{\text{distributed}}$, depending on the number of decision sites.

\begin{theorem}[Comparison Theorem]\label{thm:comparison}
Let $M_1$ be incomplete for problem $P$ and $M_2$ be complete. For any $n > 0$:
\[\text{TotalWork}(M_2, P, n) < \text{TotalWork}(M_1, P, n)\]
when considering only per-site costs (ignoring $H_{\text{central}}$).
\end{theorem}

\begin{theorem}[Amortization Threshold]\label{thm:amortization}
There exists a threshold $n^*$ such that for all $n > n^*$, the total cost of the complete model (including $H_{\text{central}}$) is strictly less than the incomplete model:
\[n^* = \frac{H_{\text{central}}(M_{\text{complete}})}{\text{SimplicityTax}(M_{\text{incomplete}}, P)}\]
Beyond $n^*$ decision sites, the complete model has lower total cost.
\end{theorem}

\begin{remark}[On the Central Cost Model]
This theorem models central cost as a scalar $H_{\text{central}}$. A more refined model might treat it as the rank of a matroid of prerequisite concepts, ensuring that different minimal learning paths have equal cardinality. The qualitative result (threshold existence) is robust to the cost model; the quantitative threshold depends on its precise formalization.
\end{remark}

\subsection{Examples}\label{sec:examples}

The conservation law applies across domains:

\begin{center}
\begin{tabular}{p{2.5cm}p{3.5cm}p{3.5cm}p{2.5cm}}
\toprule
\textbf{Domain} & \textbf{Incomplete Model} & \textbf{Complete Model} & \textbf{Tax/Site} \\
\midrule
Type Systems & Dynamic typing & Static typing & Type errors \\
Data Validation & Ad-hoc checks & Schema validation & Validation code \\
Configuration & Hardcoded values & Config management & Change sites \\
APIs & String parameters & Typed interfaces & Parse/validate \\
\bottomrule
\end{tabular}
\end{center}

In each case, the incomplete model has lower $H_{\text{central}}$ but positive $H_{\text{distributed}}$. For $n$ sites, total work is $H_{\text{central}} + n \times \text{tax}$.

\textbf{Example: Static vs. Dynamic Typing.} Dynamic typing has lower learning cost. But type errors are a per-site cost: each call site that could receive an incorrect type requires either defensive code or debugging. For $n$ call sites:
\begin{itemize}
\item Static typing: type checker verifies once; per-site cost is 0.
\item Dynamic typing: $n$ potential error sites, each with positive per-site cost.
\end{itemize}

\subsection{Connection to the Complexity Results}\label{sec:connection}

The conservation law connects to the main complexity results as follows: since identifying the required dimensions $R(P)$ is \coNP-complete (Theorem~\ref{thm:sufficiency-conp}), and since misidentification leads to a positive simplicity tax, the cost of using an incomplete model is a direct consequence of the hardness of the sufficiency problem.

\begin{corollary}
If identifying $R(P)$ is intractable, then minimizing the simplicity tax is also intractable.
\end{corollary}

\subsection{Lean 4 Formalization}\label{sec:simplicity-lean}

The theorems in this section are machine-checked in \texttt{DecisionQuotient/HardnessDistribution.lean}:

\begin{center}
\begin{tabular}{ll}
\toprule
\textbf{Theorem} & \textbf{Lean Name} \\
\midrule
Complexity Conservation & \texttt{simplicityTax\_conservation} \\
Complete Models Pay No Tax & \texttt{complete\_tool\_no\_tax} \\
Incomplete Models Pay Positive Tax & \texttt{incomplete\_tool\_positive\_tax} \\
Linear Growth & \texttt{simplicityTax\_grows} \\
Dominance & \texttt{complete\_dominates\_incomplete} \\
Amortization Threshold & \texttt{amortization\_threshold} \\
Tax Antitone w.r.t. Expressiveness & \texttt{simplicityTax\_antitone} \\
\bottomrule
\end{tabular}
\end{center}

The formalization uses \texttt{Finset $\mathbb{N}$} for dimensions. The \texttt{Model} type forms a lattice under the expressiveness ordering, with tax antitone (more expressive $\Rightarrow$ lower tax).
