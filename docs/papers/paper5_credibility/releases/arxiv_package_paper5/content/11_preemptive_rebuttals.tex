\section{Preemptive Rebuttals}\label{appendix-rebuttals}

We address anticipated objections to the credibility framework.

\subsection{Objection 1: ``Signaling theory is old---this isn't novel''}

\textbf{Objection:} ``Spence's signaling model and Crawford-Sobel's cheap talk are decades old. This paper just applies existing theory.''

\textbf{Response:} The foundations are established; the application is novel. Our contributions:

\begin{enumerate}
\item \textbf{Meta-assertions:} We apply cheap talk bounds to \emph{claims about credibility itself}---a recursive structure not analyzed in prior work
\item \textbf{Computable bounds:} We derive explicit formulas for credibility ceilings as functions of prior deception probability
\item \textbf{Leverage integration:} We connect credibility to the DOF framework, showing credibility as a form of epistemic leverage
\end{enumerate}

The theorems (Cheap Talk Bound, Magnitude Penalty, Meta-Assertion Trap) are new. The methodology is established.

\subsection{Objection 2: ``Real communication isn't Bayesian''}

\textbf{Objection:} ``Humans don't update beliefs according to Bayes' rule. The rationality assumption is unrealistic.''

\textbf{Response:} The rationality assumption provides \emph{upper bounds}. If agents deviate from Bayesian reasoning, credibility bounds may be tighter (irrational skepticism) or looser (irrational credulity). The theorems characterize what is \emph{achievable} under optimal reasoning---a ceiling, not a prediction.

For AI systems designed to be rational, the bounds are prescriptive. For human communication, they are normative benchmarks.

\subsection{Objection 3: ``Costly signals aren't always truth-dependent''}

\textbf{Objection:} ``Expensive signals can be equally costly for liars and truth-tellers. Your distinction is too clean.''

\textbf{Response:} Correct. The definition (Section 2) distinguishes:

\begin{itemize}
\item \textbf{Truth-dependent costs:} Lying is more expensive than truth-telling (e.g., maintaining consistent false memories)
\item \textbf{Truth-independent costs:} Equal cost regardless of truth value (e.g., verbose phrasing)
\end{itemize}

Expensive signals with truth-independent costs remain cheap talk despite their expense. The credibility-enhancing property comes from the \emph{differential}, not the absolute cost.

\subsection{Objection 4: ``The magnitude penalty seems wrong''}

\textbf{Objection:} ``Detailed claims are often more credible, not less. The magnitude penalty contradicts intuition.''

\textbf{Response:} The distinction is between:

\begin{itemize}
\item \textbf{Detail about the claim:} Research, evidence, specificity---these are costly signals (effort-dependent) and credibility-enhancing
\item \textbf{Detail about credibility itself:} ``I'm absolutely certain, trust me, this is verified''---these are cheap talk and subject to the magnitude penalty
\end{itemize}

The theorem applies to meta-assertions (claims about credibility), not to substantive claims. A detailed scientific argument is credibility-enhancing; a detailed assertion of trustworthiness is credibility-reducing.

\subsection{Objection 5: ``The Lean proofs are trivial''}

\textbf{Objection:} ``The Lean proofs just formalize definitions. There's no deep mathematics.''

\textbf{Response:} The value is precision, not difficulty. The proofs verify:

\begin{enumerate}
\item The cheap talk bound follows from the cost-independence definition
\item The magnitude penalty follows from the recursive structure of meta-assertions
\item The impossibility results follow from the bound composition
\end{enumerate}

Machine-checked proofs eliminate ambiguity in informal arguments. The contribution is verification, not complexity.

\subsection{Objection 6: ``This doesn't apply to AI systems''}

\textbf{Objection:} ``AI systems can be designed to be trustworthy. The cheap talk bounds don't apply to engineered systems.''

\textbf{Response:} The bounds apply to \emph{communication about} trustworthiness, not to trustworthiness itself. An AI system can be trustworthy, but its \emph{assertions} of trustworthiness are cheap talk unless backed by costly signals (audits, formal verification, track record).

The framework explains why ``I am safe and helpful'' is less credibility-enhancing than demonstrated safety and helpfulness.

\subsection{Objection 7: ``The impossibility results are too strong''}

\textbf{Objection:} ``Theorem 5.1 says no string achieves high credibility for high-magnitude claims. But some claims are believed.''

\textbf{Response:} The theorem applies to \emph{cheap talk} strings. Credibility for high-magnitude claims requires costly signals:

\begin{itemize}
\item Track record (historical accuracy)
\item Formal verification (mathematical proof)
\item Reputation stake (costly to lose)
\item Third-party attestation (independent verification)
\end{itemize}

The impossibility is for \emph{verbal assertions alone}. Credibility is achievable through costly signals, not through phrasing.

\subsection{Objection 8: ``The leverage integration is forced''}

\textbf{Objection:} ``Connecting credibility to DOF seems like a stretch. These are different concepts.''

\textbf{Response:} The connection is structural. Credibility leverage is:

\[L_C = \frac{\Delta \text{Credibility}}{\text{Signal Cost}}\]

This parallels architectural leverage:

\[L = \frac{|\text{Capabilities}|}{\text{DOF}}\]

Both measure efficiency: capability per unit of constraint. The integration shows that credibility optimization follows the same mathematical structure as architectural optimization. This unification is the theoretical contribution.

