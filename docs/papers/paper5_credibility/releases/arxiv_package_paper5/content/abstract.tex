A counterintuitive phenomenon pervades epistemic communication: emphatic assertions of trustworthiness often \emph{decrease} perceived trustworthiness. ``Trust me'' invites suspicion; excessive qualification triggers doubt rather than alleviating it. This paper provides the first formal framework explaining this phenomenon through the lens of signaling theory.

\textbf{Theorem (Cheap Talk Bound).} For any signal $s$ whose production cost is truth-independent, posterior credibility is bounded: $\Pr[C{=}1 \mid s] \leq p/(p + (1-p)q)$, where $p$ is the prior and $q$ is the mimicability of the signal. Verbal assertions---including assertions about credibility---are cheap talk and therefore subject to this bound.

\textbf{Theorem (Emphasis Penalty).} There exists a threshold $k^*$ such that for $n > k^*$ repeated assertions of claim $c$: $\partial C(c, s_{1..n})/\partial n < 0$. Additional emphasis \emph{decreases} credibility, as excessive signaling is itself informative of deceptive intent.

\textbf{Theorem (Text Credibility Bound).} For high-magnitude claims (low prior probability), no text string achieves credibility above threshold $\tau < 1$. This is an impossibility result: rephrasing cannot escape the cheap talk bound.

\textbf{Theorem (Costly Signal Escape).} Signals with truth-dependent costs---where $\text{Cost}(s \mid \text{false}) > \text{Cost}(s \mid \text{true})$---can achieve arbitrarily high credibility as the cost differential increases. Machine-checked proofs are maximally costly signals: producing a compiling proof of a false theorem has infinite cost.

These results integrate with the leverage framework (Paper 3): credibility leverage $L_C = \Delta C / \text{Signal Cost}$ is maximized by minimizing cheap talk and maximizing costly signal exposure. The theorems are formalized in Lean 4.

\textbf{Keywords:} signaling theory, cheap talk, credibility, Bayesian epistemology, costly signals, formal verification, Lean 4