\section{Related Work}\label{related-work}

\textbf{Signaling Theory:} Spence (1973) \cite{spence1973job} introduced costly signaling in
job markets. Zahavi (1975) \cite{zahavi1975mate} applied it to biology (handicap principle).
Akerlof (1970) \cite{akerlof1970market} established the foundational role of asymmetric information 
in market collapse. We formalize and extend to text-based communication.

\textbf{Cheap Talk:} Crawford \& Sobel (1982) \cite{crawford1982strategic} analyzed cheap talk in
game theory. Farrell (1987) \cite{farrell1987cheap} and Farrell \& Rabin (1996) \cite{farrell1996cheap} 
further characterized the limits of unverified communication. We prove explicit bounds on credibility shift.

\textbf{Epistemic Logic:} Hintikka (1962) \cite{hintikka1962knowledge}, Fagin et al.~(1995) \cite{fagin1995reasoning}
formalized knowledge and belief. We add signaling structure.

\textbf{Bayesian Persuasion:} Kamenica \& Gentzkow (2011) \cite{kamenica2011bayesian} studied
optimal information disclosure. Our impossibility results complement
their positive results.

\textbf{Social Epistemology:} Goldman (1999) \cite{goldman1999social} and Hardwig (1991) \cite{hardwig1991trust}
studied the social dimensions of knowledge. Our dual truth framework extends this to include
ego-driven truth as a complement to epistemic truth.

\textbf{Cognitive Dissonance:} Festinger (1957) \cite{festinger1957theory} introduced cognitive dissonance theory,
which explains how individuals resolve conflicts between beliefs and actions. Our coherence measure
quantifies this dissonance as a gap between epistemic and ego-driven truth.

\textbf{Dual Process Theories:} Kahneman (2011) \cite{kahneman2011thinking} and Evans (2003) \cite{evans2003in two}
distinguished between fast, intuitive thinking (System 1) and slow, deliberate thinking (System 2).
Our dual truth framework aligns with this distinction: ego-driven truth often operates through System 1,
while epistemic truth requires System 2 reasoning.

\begin{center}\rule{0.5\linewidth}{0.5pt}\end{center}

