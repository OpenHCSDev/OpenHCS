\section{Impossibility Results}\label{impossibility-results}

\subsection{The Text Credibility
Bound}\label{the-text-credibility-bound}

\begin{theorem}[Text Credibility Bound]\label{thm:text-bound}
For any text string $T$ (memory content, assertion, etc.) and high-magnitude
claim $c$ with $M(c) > M^*$ (i.e., prior $p < e^{-M^*}$):
\[\Pr[c \mid T] < \tau\]
where $\tau < 1$ is determined by the mimicability $q$ and $M^*$.
No text achieves full credibility for exceptional claims.
\end{theorem}

\begin{proof}
Text is cheap talk (production cost independent of truth). Apply
Theorem~\ref{thm:cheap-talk-bound} with prior $p = e^{-M^*}$ and mimicability $q$:
\[
\tau = \frac{p}{p + (1-p)q} = \frac{e^{-M^*}}{e^{-M^*} + (1 - e^{-M^*})q}
\]
For $M^*$ large (low prior probability), $\tau \to 0$ regardless of $q > 0$.
\end{proof}

\textbf{Corollary 5.2 (Memory Iteration Futility).} No rephrasing of
memory content can achieve credibility above \(\tau\) for high-magnitude
claims. Iteration on text is bounded in effectiveness.

\textbf{Interpretation:} This is why we couldn't solve the credibility
problem by editing memory text. The \emph{structure} of the problem
(text is cheap talk, claims are high-magnitude) guarantees bounded
credibility regardless of phrasing.

\subsection{Optimal Strategies}\label{optimal-strategies}

\textbf{Theorem 5.3 (Optimal Credibility Strategy).} For high-magnitude
claims, the credibility-maximizing strategy is: 1. Minimize cheap talk
(reduce emphasis, meta-assertions) 2. Maximize costly signal exposure
(show the work, provide proofs) 3. Enable real-time demonstration
(costly to fake)

\begin{center}\rule{0.5\linewidth}{0.5pt}\end{center}

