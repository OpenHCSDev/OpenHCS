%%%%%%%%%%%%%%%%%%%%%%%%%%%%%%%%%%%%%%%%%%%%%%%%%%%%%%%%%%%%%%%%%%%%%%%%%%%%%%%
\section{Conclusion}\label{sec:conclusion}
%%%%%%%%%%%%%%%%%%%%%%%%%%%%%%%%%%%%%%%%%%%%%%%%%%%%%%%%%%%%%%%%%%%%%%%%%%%%%%%

Formality = Universality. This is not metaphor. It's equivalence.

The informal is local. It requires shared context to mean anything. Remove the context, meaning collapses.

The formal crosses all boundaries because it eliminates ambiguity. A compiled proof is the same proof for all agents. That's the only thing that scales across cultures, centuries, and cognitive architectures.

\begin{enumerate}
\item \textbf{Formal $\leftrightarrow$ Universal}: Machine-checked proofs compile everywhere because formality eliminates interpretation.
\item \textbf{Opinion = Intractable Objectivity}: ``Opinions'' are truths with intractable model-finding. When coordinates become identifiable, opinion evaporates.
\item \textbf{Universe Denial Incoherent}: ``In your model'' objections require witnesses. No witnesses exist for open structures with compiled proofs.
\item \textbf{Objections Without Witnesses = Cheap Talk}: By Paper 5, bounded credibility.
\item \textbf{Coherent Agents Respect Verifiers}: Maintain verified claims, concede refuted claims.
\item \textbf{Incoherence $\neq$ Evil}: Computational limitation. Response: type error, not contempt.
\end{enumerate}

\begin{quote}
\textbf{Mathematics is the universal language because it's the only thing that compiles everywhere.}
\end{quote}

