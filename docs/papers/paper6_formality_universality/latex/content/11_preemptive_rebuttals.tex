\section{Preemptive Rebuttals}\label{appendix-rebuttals}

We address anticipated objections to the formal-universal equivalence.

\subsection{Objection 1: ``Formality excludes important truths''}

\textbf{Objection:} ``Many important truths---ethical, aesthetic, experiential---cannot be formalized. Your framework dismisses them.''

\textbf{Response:} The claim is not that informal propositions are false or unimportant, but that they are \emph{local}. A proposition is local if its usefulness depends on shared context (language, culture, assumptions).

Formal propositions are universal precisely because they don't require shared context. Both have value; they have different scope:

\begin{itemize}
\item \textbf{Local:} ``This painting is beautiful'' (requires aesthetic context)
\item \textbf{Universal:} ``$2 + 2 = 4$'' (compiles for any agent)
\end{itemize}

The framework distinguishes scope, not importance.

\subsection{Objection 2: ``The equivalence is circular''}

\textbf{Objection:} ``You define formal as 'verifiable by any agent' and universal as 'useful to any agent.' These are the same thing.''

\textbf{Response:} The definitions are independent:

\begin{itemize}
\item \textbf{Formal:} There exists a verifier $V$ such that any agent can run $V$ and $V$ decides $P$
\item \textbf{Universal:} For all agents, $P$ is useful
\end{itemize}

The theorem proves these coincide. This is substantive: it's not obvious that verifiability implies usefulness or vice versa. The proof shows that agent-independence (universality) requires explicit rules (formality), and explicit rules (formality) enable agent-independence (universality).

\subsection{Objection 3: ``Mathematics requires training''}

\textbf{Objection:} ``Mathematics isn't universal---it requires years of training to understand. A child can't verify a proof.''

\textbf{Response:} Training enables \emph{running} the verifier, not \emph{accepting} the result. The distinction:

\begin{itemize}
\item \textbf{Running:} Executing the verification procedure (requires training)
\item \textbf{Accepting:} Trusting the output once verified (universal)
\end{itemize}

A Song dynasty mathematician and a modern computer scientist may use different notation, but both can verify the same proof. The verification is universal; the pedagogy is local.

Moreover, machine verification eliminates the training requirement. Lean doesn't need to ``understand'' mathematics---it executes the verifier. This is the ultimate universality: even non-mathematical agents can verify.

\subsection{Objection 4: ``This is logical positivism''}

\textbf{Objection:} ``This sounds like logical positivism---the claim that non-verifiable statements are meaningless. That view was refuted.''

\textbf{Response:} Logical positivism claimed informal statements are \emph{meaningless}. We claim they are \emph{local}. The distinction is crucial:

\begin{itemize}
\item \textbf{Positivism:} ``God exists'' is meaningless (no verification procedure)
\item \textbf{Our claim:} ``God exists'' is local (useful only to agents sharing theological context)
\end{itemize}

Local propositions can be true, useful, and important. They just don't compile everywhere. This is a scope claim, not a meaning claim.

\subsection{Objection 5: ``The Lean proofs are trivial''}

\textbf{Objection:} ``The proofs just formalize definitions. There's no deep mathematics.''

\textbf{Response:} The value is precision. The informal argument ``formal implies universal'' becomes a machine-checked derivation. The proofs verify:

\begin{enumerate}
\item The forward direction (formal $\to$ universal) follows from verifier universality
\item The backward direction (universal $\to$ formal) follows from agent-independence requiring explicit rules
\item The equivalence is symmetric and transitive
\end{enumerate}

Trivial proofs that compile are more valuable than deep proofs with errors.

\subsection{Objection 6: ``Opinion complexity is not well-defined''}

\textbf{Objection:} ``You claim opinions have complexity, but complexity is a property of algorithms, not beliefs.''

\textbf{Response:} Opinion complexity is defined as the minimum description length of the context required to make the opinion useful. This is well-defined:

\begin{itemize}
\item \textbf{Low complexity:} ``$2 + 2 = 4$'' (no context required)
\item \textbf{High complexity:} ``This wine is excellent'' (requires shared aesthetic, cultural, experiential context)
\end{itemize}

The definition uses Kolmogorov complexity of the context, which is a standard measure.

\subsection{Objection 7: ``Universe denial is too strong''}

\textbf{Objection:} ``Theorem 3.1 says rejecting formal proofs requires denying the universe. This is hyperbolic.''

\textbf{Response:} The theorem is precise. To reject a machine-checked proof, you must deny one of:

\begin{enumerate}
\item The verifier is correct (deny mathematics)
\item The machine executed correctly (deny physics)
\item The output is what you observed (deny perception)
\end{enumerate}

Each denial has unbounded cost---it undermines all reasoning in that domain. ``Denying the universe'' is shorthand for this unbounded cost. The theorem is not hyperbolic; it's a precise characterization of the epistemic cost of rejection.

\subsection{Objection 8: ``Coherent agency is too restrictive''}

\textbf{Objection:} ``Your definition of coherent agency excludes agents with inconsistent beliefs. Most humans are incoherent.''

\textbf{Response:} Coherent agency is a normative ideal, not a descriptive claim. The theorems characterize what \emph{coherent} agents must accept. Incoherent agents can reject anything---but at the cost of incoherence.

The framework provides a standard: if you want to be coherent, you must accept formal proofs. If you reject them, you are (by definition) incoherent. This is not a criticism of humans; it's a characterization of the epistemic landscape.

