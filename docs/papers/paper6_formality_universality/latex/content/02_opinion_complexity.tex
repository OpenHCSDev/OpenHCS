%%%%%%%%%%%%%%%%%%%%%%%%%%%%%%%%%%%%%%%%%%%%%%%%%%%%%%%%%%%%%%%%%%%%%%%%%%%%%%%
\section{Opinion as Complexity Class}\label{sec:opinion}
%%%%%%%%%%%%%%%%%%%%%%%%%%%%%%%%%%%%%%%%%%%%%%%%%%%%%%%%%%%%%%%%%%%%%%%%%%%%%%%

\subsection{The Reframe}

``Opinion'' is the name we give to underdetermined computation. The formal structure exists. The model is just too high-dimensional to fit in a conversation, a lifetime, a civilization.

\begin{definition}[Opinion]
\begin{verbatim}
def opinion (p : Prop) : Prop :=
  exists model : Model, decidable_in model p /\\
  notcomputable_in_poly_time (find model)
\end{verbatim}
\end{definition}

``Opinions'' are propositions where:
\begin{enumerate}
\item A model exists that decides them
\item Finding/specifying that model is intractable
\end{enumerate}

It's not that truth doesn't exist. It's that the sufficient coordinate set (Paper 4) is too large to identify. So we call it ``opinion'' and move on.

\subsection{When Opinion Evaporates}

Sometimes the model is small. Sometimes the sufficient coordinates are few. Sometimes it compiles in seconds. Then ``opinion'' evaporates. There's just the answer.

\begin{theorem}[Opinion is Complexity Class]
\begin{verbatim}
theorem opinion_is_complexity_class :
  is_opinion p <->
  (exists truth_value p) /\\
  (complexity (determine p) > available_resources)
\end{verbatim}
\end{theorem}

The boundary between ``opinion'' and ``fact'' is not metaphysical. It's computational. As resources increase or models simplify, opinion becomes fact.

\subsection{Implications for Disagreement}

Someone holding an ``opinion'' is not irrational. They're running a heuristic on intractable input. The heuristic usually works. It doesn't here. That's not moral failure. That's computational limitation.

