%%%%%%%%%%%%%%%%%%%%%%%%%%%%%%%%%%%%%%%%%%%%%%%%%%%%%%%%%%%%%%%%%%%%%%%%%%%%%%%
\section{Introduction}\label{sec:introduction}
%%%%%%%%%%%%%%%%%%%%%%%%%%%%%%%%%%%%%%%%%%%%%%%%%%%%%%%%%%%%%%%%%%%%%%%%%%%%%%%

\subsection{The Central Equivalence}

\begin{definition}[Formal Proposition]
A proposition $P$ is \emph{formal} if there exists a verifier $V$ such that any agent can run $V$ and $V$ decides $P$:
\begin{verbatim}
def formal (P : Prop) : Prop :=
  exists V : Verifier, forall agent, agent.can_run V /\\ V.decides P
\end{verbatim}
\end{definition}

\begin{definition}[Universal Proposition]
A proposition $P$ is \emph{universal} if it is useful to all agents:
\begin{verbatim}
def universal (P : Prop) : Prop :=
  forall agent, useful P agent
\end{verbatim}
\end{definition}

\begin{theorem}[Formal-Universal Equivalence]\label{thm:formal-universal}
\begin{verbatim}
theorem formal_iff_universal : formal P <-> universal P
\end{verbatim}
\end{theorem}

\textbf{Proof sketch.}

\emph{Forward direction}: If formal, then any agent can run the verifier. The verification is independent of language, culture, time, location, beliefs, status. A Song dynasty mathematician, an American engineer, a 23rd century AI---all get the same answer. That's universality.

\emph{Backward direction}: If universal, must be agent-independent. Agent-independence requires explicit rules that don't depend on interpretation. Explicit rules that decide propositions = verifier. That's formality.

\begin{corollary}[Informal is Local]
The informal is local by definition. It requires shared context, shared assumptions, shared language, shared culture. Remove any of these, usefulness collapses. Therefore not universal.
\end{corollary}

\begin{quote}
\textbf{Mathematics is the universal language not because it's ``pure'' or ``beautiful''---because it's the only thing that compiles everywhere.}
\end{quote}

\subsection{Connection to Previous Work}

This paper builds on and synthesizes results from the preceding papers:

\begin{itemize}
\item \textbf{Paper 1} (Typing Discipline): Machine-checked proofs of nominal dominance---formal proofs that compile for any reader
\item \textbf{Paper 4} (Decision Quotient): Computational hardness of coordinate identification---when ``opinion'' becomes ``fact''
\item \textbf{Paper 5} (Credibility): Cheap talk bounds on assertions without costly signals---the epistemology of verification
\end{itemize}

This paper synthesizes: formal proofs are costly signals (Paper 5) because compilation is truth-dependent cost. ``In your model'' objections are cheap talk. The asymmetry is total.

\subsection{Anticipated Objections}\label{sec:objection-summary}

Before proceeding, we address objections readers are likely forming. Each is refuted in detail in Appendix~\ref{appendix-rebuttals}.

\paragraph{``Formality excludes important truths.''}
The claim is not that informal propositions are false, but that they are \emph{local}---useful only to agents sharing context. Formal propositions are universal precisely because they don't require shared context. Both have value; they have different scope.

\paragraph{``The formal-universal equivalence is circular.''}
The definitions are independent. Formality is defined by verifier existence; universality is defined by agent-independence. The theorem proves they coincide. This is a substantive claim, not a tautology.

\paragraph{``Mathematics isn't universal---it requires training.''}
Training enables \emph{running} the verifier, not \emph{accepting} the result. A Song dynasty mathematician and a modern computer scientist may use different notation, but both can verify the same proof. The verification is universal; the pedagogy is local.

\paragraph{``This is just logical positivism repackaged.''}
Logical positivism claimed informal statements are \emph{meaningless}. We claim they are \emph{local}. The distinction is crucial: local propositions can be true, useful, and important---they just don't compile everywhere.

\paragraph{``The Lean proofs are trivial.''}
The proofs formalize the equivalence structure. The value is precision: the informal argument ``formal implies universal'' becomes a machine-checked derivation. Trivial proofs that compile are more valuable than deep proofs with errors.

\medskip
\noindent\textbf{If you have an objection not listed above,} check Appendix~\ref{appendix-rebuttals} before concluding it has not been considered.

