The axis orthogonality paper (Paper 1) and the Single Source of Truth paper (Paper 2) establish necessary conditions for coherent software architecture: axis orthogonality and DOF = 1 per fact. Among architectures satisfying these constraints, this paper proves the optimization criterion is computable and decidable: maximize leverage $L = |\text{Capabilities}|/\text{DOF}$.

The axis orthogonality paper proves minimal complete axis sets are orthogonal for classification systems. The SSOT paper proves DOF = 1 is necessary and sufficient for epistemic coherence. These establish the constraint space for coherent architectures. This paper derives the optimization criterion within that space.

We prove seven theorems (machine-checked, 0 sorries). Error Independence (Theorem~3.1): axis orthogonality implies statistical independence of errors across DOF, derived from matroid theory. Error Compounding (Theorem~3.3): for a system with $n$ DOF and per-component error rate $p$, system error probability is $P_{\text{error}}(n) = 1 - (1-p)^n$, following from the SSOT paper's coherence theorem. DOF-Reliability Isomorphism (Theorem~3.4): an architecture with $n$ DOF is isomorphic to a series reliability system with $n$ components, preserving failure probability ordering. Leverage-Error Tradeoff (Theorem~4.1): for architectures with equal capabilities, higher leverage strictly implies lower error probability. Modification Complexity Gap (Theorem~4.2): for architectures with equal capabilities, the expected modification ratio equals the DOF ratio, growing unbounded. Optimal Architecture (Theorem~4.3): given requirements $R$, architecture $A^*$ minimizes error probability iff it satisfies feasibility and maximality constraints, yielding a decidable criterion. Metaprogramming Dominance (Theorem~4.4): for metaprogramming architectures with DOF = 1 and unbounded derivations, leverage approaches infinity.

The three papers form a complete framework. The axis orthogonality paper's orthogonality result enables our error independence result (Theorem~3.1). The SSOT paper's coherence requirement enables our error compounding result (Theorem~3.3). Given constraints from the prior papers, we prove leverage maximization minimizes error probability (Theorem~4.1).

Three instances demonstrate integration. For SSOT vs scattered architecture with $n$ use sites, SSOT achieves $L = c$ while scattered achieves $L = c/n$, yielding unbounded leverage ratio. For nominal vs duck typing, the axis orthogonality paper's four additional capabilities (provenance, identity, enumeration, conflict resolution) yield $L_{\text{nominal}} = (c + 4)/d > c/d = L_{\text{duck}}$ with equal DOF. For microservices vs monolith with $n$ services of $c$ capabilities each, monolith achieves $L = nc$ while microservices achieve $L = c$, yielding $n\times$ leverage advantage.

Given coherence as a requirement, architectural choice becomes deterministic. The optimization criterion $L(A) = |\text{Cap}(A)|/\text{DOF}(A)$ is decidable, the optimal architecture is unique up to isomorphism, and preference dissolves into mathematical necessity.

OpenHCS PR \#44 validates the framework: migrating from duck typing (47 scattered checks, DOF = 47) to nominal ABC (1 definition, DOF = 1) increased leverage $47\times$ and improved error localization from $\Omega(n)$ to $O(1)$.

Lean 4 formalization: 741 lines, 35 theorems, 0 \texttt{sorry}. Applies theorems established in the axis orthogonality paper and the SSOT paper to derive leverage optimization criterion.

Keywords: software architecture, leverage, degrees of freedom, epistemic coherence, reliability theory, formal methods, optimization