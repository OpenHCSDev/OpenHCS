\section{Instances}\label{instances}

Papers 1--3 already prove theoretical uniqueness for specific domains. We show how 
they instantiate the general pattern.

\subsection{Paper 1: Axis Orthogonality}\label{paper1-instance}

\textbf{Domain:} Dataset with $n$ points in $\mathbb{R}^d$, finite attribute values.

\textbf{Queries:} 
\begin{itemize}
\item "Are attributes $i$ and $j$ independent?"
\item "What is the covariance between attributes?"
\item "Which coordinate system minimizes redundancy?"
\end{itemize}

\textbf{Implementation:} All $n \times d$ coordinate values.

\textbf{Minimal Theory:} Orthogonal coordinate axes.

\textbf{Uniqueness:} Paper 1 proves that for finite discrete attributes, orthogonal 
axes are the unique minimal coordinate system. Any non-orthogonal system introduces 
redundant parameters (correlations between axes).

\textbf{Instance of Theorem~\ref{thm:unique-minimal-theory}:} Orthogonal axes are 
unique minimal theory for attribute independence queries.

\textbf{Instance of Theorem~\ref{thm:computability-from-queries}:} Given queries about 
independence, the algorithm extracts the unique orthogonal basis.

\subsection{Paper 2: Single Source of Truth}\label{paper2-instance}

\textbf{Domain:} Multi-scale representation system with coherence requirements.

\textbf{Queries:}
\begin{itemize}
\item "What is the value at scale $s$ and position $p$?"
\item "Are scales $s_1$ and $s_2$ coherent?"
\item "How many degrees of freedom does the system have?"
\end{itemize}

\textbf{Implementation:} Complete specification of all scale-specific values.

\textbf{Minimal Theory:} Single source of truth with projection operators.

\textbf{Uniqueness:} Paper 2 proves DOF = 1 for coherent systems. The minimal theory 
has one authoritative representation; all others are projections. Any multi-source 
system either violates coherence or contains redundancy.

\textbf{Instance of Theorem~\ref{thm:unique-minimal-theory}:} SSOT is unique minimal 
theory for coherent multi-scale queries.

\textbf{Instance of Theorem~\ref{thm:compression-necessity}:} Infinite scale queries 
require compression into single source.

\subsection{Paper 3: Leverage Uniqueness}\label{paper3-instance}

\textbf{Domain:} Statistical dataset with $n$ points, selection criteria needed.

\textbf{Queries:}
\begin{itemize}
\item "Which point has maximum influence on model fit?"
\item "What is the optimal removal criterion?"
\item "How does removing point $i$ affect uncertainty?"
\end{itemize}

\textbf{Implementation:} All point coordinates and statistical relationships.

\textbf{Minimal Theory:} Weighted leverage scores.

\textbf{Uniqueness:} Paper 3 proves weighted leverage is the unique optimization 
criterion under statistical invariance. Any alternative criterion either fails some 
optimality query or adds unnecessary parameters.

\textbf{Instance of Theorem~\ref{thm:unique-minimal-theory}:} Leverage is unique 
minimal theory for influence queries.

\subsection{The General Pattern}\label{general-pattern}

All three papers follow identical structure:
\begin{enumerate}
\item Define domain $D$ and query space $\text{Queries}(D)$
\item Show minimal theory $T^*$ answers all queries
\item Prove any alternative theory either incomplete or redundant
\item Conclude $T^*$ is unique
\end{enumerate}

This is not coincidence---it is Theorem~\ref{thm:unique-minimal-theory} applied.

\begin{center}\rule{0.5\linewidth}{0.5pt}\end{center}

