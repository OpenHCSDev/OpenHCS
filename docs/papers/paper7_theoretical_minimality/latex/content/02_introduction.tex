\section{Introduction}\label{introduction}

Scientific theories compress observations into minimal explanatory structures. 
The Standard Model reduces particle physics to 19 parameters. General relativity 
compresses gravitational phenomena into a single field equation. This compression 
is not merely aesthetic---it is computational necessity.

This paper proves a fundamental result in formal epistemology, grounded in the 
\emph{One-Universe Framework (OUF)} where axioms are definitions and truth is 
absolute:

\begin{quote}
\textbf{For any finite domain, the minimal theory that answers all possible queries 
is unique up to isomorphism and computable from the query space.}
\end{quote}

\subsection{The Uniqueness Question}\label{the-uniqueness-question}

Can multiple distinct theories equally explain the same phenomena? Philosophical 
pluralists argue yes---equivalent formulations reflect convention rather than 
discovery. We prove the opposite.

Within the One-Universe Framework (formalized in Section~\ref{one-universe-framework}), 
truth is absolute, not model-relative. Axioms are definitional statements about what 
terms mean in the mathematical universe $\mathcal{U}$, not assumptions about multiple 
possible models. This foundational commitment has a profound consequence: minimal 
theories are unique.

\textbf{Definition:} A \emph{theory} is a function $T: Q \to A$ mapping queries 
to answers for a domain $D$. A theory is \emph{minimal} if no proper subset of 
$T$ answers all queries in $\text{Queries}(D)$.

\textbf{Theorem 1.1 (Unique Minimal Theory):} For any finite domain $D$, there 
exists exactly one minimal theory $T^*$ up to isomorphism. All other theories 
either:
\begin{enumerate}
\item Fail to answer some query in $\text{Queries}(D)$, or
\item Contain redundant structure not required by any query
\end{enumerate}

\textbf{Foundational Grounding:} This uniqueness follows from the One-Universe 
Framework's collapse of model-theoretic independence. Since truth is absolute 
and axioms are definitions, the minimal set of parameters required to answer 
all queries is uniquely determined by the intrinsic dimension of the domain. 
There is no room for ``equivalent but distinct'' theories---minimality and 
orthogonality collapse into a single constraint (Collapse Result 1, Section~\ref{one-universe-framework}).

This is not mere mathematical pedantry. Papers 1--3 already demonstrate this 
pattern in concrete domains:

\subsection{Instances in Prior Work}\label{instances-in-prior-work}

\textbf{Paper 1 (Axis Orthogonality):} Proves that for any finite dataset with 
$n$ distinct values per attribute, the unique minimal theory is orthogonal 
coordinate axes. All queries about attribute independence are answered by checking 
axis alignment. Any non-orthogonal system introduces redundant parameters.

\textbf{Paper 2 (SSOT):} Proves that coherent multi-scale representations have 
exactly one degree of freedom. The minimal theory is the single source of truth. 
All scale-specific views are computable projections. Any additional structure 
violates coherence or introduces redundancy.

\textbf{Paper 3 (Leverage):} Proves that weighted leverage is the unique 
optimization criterion for finite datasets under statistical invariance. All 
queries about optimal point selection are answered by leverage scores. Any 
alternative criterion either fails some query or contains unnecessary parameters.

These are not accidents. They instantiate a general pattern.

\subsection{From Compression to Computation}\label{from-compression-to-computation}

Why does minimality entail uniqueness? The key insight connects compression to 
computation.

\textbf{Theorem 1.2 (Compression Necessity):} For any domain $D$ with infinite 
query space $|\text{Queries}(D)| = \infty$, any theory $T$ that answers all 
queries must satisfy $|T| < |\text{Implementation}|$. Direct lookup tables are 
uncomputable.

Compression forces structure. Structure determines the theory. Once we know which 
queries must be answered, the minimal structure is fixed.

\textbf{Theorem 1.3 (Computability from Queries):} The minimal theory $T^*$ is a 
computable function of the query space: $T^* = f(\text{Queries}(D))$ where $f$ 
is algorithmic.

This has profound implications: discovering a theory is not creative interpretation 
but mechanical extraction. Given a domain and its query space, the minimal theory 
is determined.

\subsection{Contributions}\label{contributions}

\begin{enumerate}
\def\labelenumi{\arabic{enumi}.}
\item \textbf{Formal Framework (Section~\ref{foundations}):} Rigorous definitions of 
theories, implementations, query spaces, domains, and minimality.

\item \textbf{Uniqueness Theorems (Section~\ref{uniqueness-theorems}):}
\begin{itemize}
\item Theorem 3.1: Unique Minimal Theory (minimal $T^*$ unique up to isomorphism)
\item Theorem 3.2: Compression Necessity (infinite queries require $|T| < |I|$)
\item Theorem 3.3: Computability from Queries ($T^* = f(\text{Queries}(D))$)
\end{itemize}

\item \textbf{Anti-Pluralism Results (Section~\ref{anti-pluralism}):}
\begin{itemize}
\item Theorem 4.1: Incoherence of Pluralism (multiple minimal theories impossible)
\item Corollary 4.2: Convention vs Discovery (isomorphisms are relabelings)
\end{itemize}

\item \textbf{Instances (Section~\ref{instances}):} Demonstrate that Papers 1--3 
are concrete realizations of the general uniqueness pattern.

\item \textbf{Lean Formalization (Appendix~\ref{lean-formalization}):} Complete 
machine-verified proofs in Lean 4 of all theorems.
\end{enumerate}

\subsection{Philosophical Implications}\label{philosophical-implications}

This result challenges epistemic pluralism. If minimal theories are unique, then:

\begin{itemize}
\item \textbf{Scientific realism:} Theories converge because reality has unique 
minimal structure, not because scientists impose arbitrary conventions.
\item \textbf{Theory choice:} Selecting simpler theories is not aesthetic 
preference but computational necessity.
\item \textbf{Underdetermination:} Apparent theoretical alternatives either answer 
different queries or contain redundant structure.
\end{itemize}

The proof is constructive. We show how to compute $T^*$ from $\text{Queries}(D)$.

\subsection{Anticipated Objections}\label{sec:objection-summary}

Before proceeding, we address objections readers are likely forming. Each is refuted in detail in Appendix~\ref{appendix-rebuttals}.

\paragraph{``Uniqueness up to isomorphism is trivial.''}
Isomorphism is not relabeling. Two theories are isomorphic if they have the same structure---the same query-answer relationships. The theorem says there is exactly one such structure. Different notations for the same theory are not ``different theories.''

\paragraph{``This assumes the One-Universe Framework.''}
Yes. The uniqueness result depends on truth being absolute, not model-relative. Within model-theoretic pluralism, multiple theories can be ``equally true'' in different models. The OUF collapses this: axioms are definitions, and truth is determined by the mathematical universe $\mathcal{U}$.

\paragraph{``Finite domains are too restrictive.''}
The finite domain assumption enables constructive proofs. Extension to infinite domains requires additional machinery (compactness, well-foundedness). The finite case captures the essential insight: minimality determines uniqueness.

\paragraph{``This contradicts Quine's underdetermination thesis.''}
Quine's thesis applies to empirical theories with observational equivalence. Our framework applies to formal theories with query equivalence. The distinction: empirical theories can have observationally equivalent but structurally distinct formulations; formal theories cannot have query-equivalent but structurally distinct minimal formulations.

\paragraph{``The Lean proofs are trivial.''}
The proofs formalize the uniqueness structure. The value is precision: the informal argument ``minimal implies unique'' becomes a machine-checked derivation. Trivial proofs that compile are more valuable than deep proofs with errors.

\medskip
\noindent\textbf{If you have an objection not listed above,} check Appendix~\ref{appendix-rebuttals} before concluding it has not been considered.

\begin{center}\rule{0.5\linewidth}{0.5pt}\end{center}

