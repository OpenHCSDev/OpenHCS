Scientific theories are not truth itself, but minimal descriptions of regularities. This paper formalizes the epistemological status of theories as design patterns: abstract, computable structures that compress observations into explanations. We prove that for any domain, minimal theories exist, are unique up to isomorphism, and are strictly smaller than the implementations they explain.

\textbf{Theorem (Theory as Query-Answer Mapping).} A theory $T$ for domain $D$ is a function $T: \text{Queries}(D) \to \text{Answers}$ satisfying completeness, minimality, and consistency. The theory is the minimal description sufficient to answer all queries in the domain.

\textbf{Theorem (Compression Necessity).} For any domain $D$ with infinite query space, any finite theory $T$ satisfies $|T| < |I|$ where $I$ is the full implementation. Theories compress implementations by eliminating irrelevant details.

\textbf{Theorem (Unique Minimal Theory).} For any domain $D$, there exists a unique minimal theory $T^*$ (up to isomorphism) such that (1) $T^*$ is complete for $D$, and (2) $|T^*| = \min\{|T| : T \text{ is complete for } D\}$. Multiple equally valid theories cannot coexist for the same domain.

\textbf{Theorem (Computability from Queries).} The minimal theory $T^*$ for domain $D$ is computable from the query set: $T^* = f(\text{Queries}(D))$ where $f$ is a computable function. This proves theories are discovered, not chosen.

\textbf{Theorem (Incoherence of Theoretical Pluralism).} For domain $D$ with unique minimal theory $T^*$, claiming ``multiple equally valid theories exist'' instantiates $P \land \neg P$. Uniqueness entails $\neg\exists$ alternatives; pluralism presupposes $\exists$ alternatives.

\textbf{Connection to Prior Work.} Papers 1--3 already prove theoretical uniqueness for specific domains. Paper 1 proves axis orthogonality is the unique minimal representation for classification. Paper 2 proves DOF = 1 is the unique coherence condition. Paper 3 proves $L = |\text{Capabilities}|/\text{DOF}$ is the unique optimization criterion. This paper extracts the general pattern.

The theorems establish a formal epistemology where theories are mathematical objects with provable properties. ``Design pattern vs implementation'' is not metaphor but formal correspondence. All proofs mechanized in Lean 4.

\textbf{Keywords:} formal epistemology, theory minimality, Kolmogorov complexity, theoretical uniqueness, philosophy of science, Lean 4