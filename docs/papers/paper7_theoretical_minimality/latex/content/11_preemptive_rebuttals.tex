\section{Preemptive Rebuttals}\label{appendix-rebuttals}

We address anticipated objections to the theoretical minimality framework.

\subsection{Objection 1: ``Uniqueness up to isomorphism is trivial''}

\textbf{Objection:} ``Saying theories are unique 'up to isomorphism' is vacuous. Any two things are isomorphic under some mapping.''

\textbf{Response:} Isomorphism is not arbitrary relabeling. Two theories are isomorphic if they have the same structure---the same query-answer relationships, the same dependencies, the same computational properties.

The theorem says there is exactly one such structure for any finite domain. Different notations for the same theory (e.g., matrix vs index notation) are not ``different theories''---they are presentations of the same theory.

The uniqueness is substantive: given a domain and query space, the minimal theory is determined. There is no room for ``equally good but structurally distinct'' alternatives.

\subsection{Objection 2: ``This assumes the One-Universe Framework''}

\textbf{Objection:} ``The uniqueness result depends on your One-Universe Framework. Other foundational frameworks allow pluralism.''

\textbf{Response:} Yes. The uniqueness result is conditional on the OUF. Within model-theoretic pluralism, multiple theories can be ``equally true'' in different models.

The OUF collapses this: axioms are definitions (not assumptions), and truth is determined by the mathematical universe $\mathcal{U}$. This is a foundational commitment, not a hidden assumption. The paper is explicit about this grounding.

The contribution is showing that \emph{if} you accept the OUF, \emph{then} minimal theories are unique. Readers who reject the OUF may reject the conclusion---but they must reject the framework, not the derivation.

\subsection{Objection 3: ``Finite domains are too restrictive''}

\textbf{Objection:} ``Real domains are infinite. The finite domain assumption limits applicability.''

\textbf{Response:} The finite domain assumption enables constructive proofs. Extension to infinite domains requires additional machinery:

\begin{itemize}
\item \textbf{Compactness:} Infinite domains with finite query spaces reduce to finite cases
\item \textbf{Well-foundedness:} Infinite domains with well-founded query orderings admit inductive proofs
\item \textbf{Approximation:} Infinite domains can be approximated by finite subdomains
\end{itemize}

The finite case captures the essential insight: minimality determines uniqueness. The infinite case is future work, not a refutation.

\subsection{Objection 4: ``This contradicts Quine's underdetermination''}

\textbf{Objection:} ``Quine proved that theories are underdetermined by evidence. Your uniqueness result contradicts this.''

\textbf{Response:} Quine's thesis applies to \emph{empirical} theories with \emph{observational} equivalence. Our framework applies to \emph{formal} theories with \emph{query} equivalence.

The distinction:
\begin{itemize}
\item \textbf{Empirical:} Theories can be observationally equivalent but structurally distinct (different ontologies, same predictions)
\item \textbf{Formal:} Theories cannot be query-equivalent but structurally distinct (same queries $\Rightarrow$ same minimal structure)
\end{itemize}

Quine's underdetermination concerns the gap between observation and theory. Our uniqueness concerns the relationship between queries and minimal structure. These are different claims about different domains.

\subsection{Objection 5: ``The Lean proofs are trivial''}

\textbf{Objection:} ``The proofs just formalize definitions. There's no deep mathematics.''

\textbf{Response:} The value is precision. The informal argument ``minimal implies unique'' becomes a machine-checked derivation. The proofs verify:

\begin{enumerate}
\item Minimality is well-defined (no proper subset answers all queries)
\item Uniqueness follows from minimality (two minimal theories must be isomorphic)
\item Computability follows from finiteness (the minimal theory is extractable)
\end{enumerate}

Trivial proofs that compile are more valuable than deep proofs with errors.

\subsection{Objection 6: ``Compression necessity is obvious''}

\textbf{Objection:} ``Of course infinite query spaces require compression. This is just information theory.''

\textbf{Response:} The theorem is not that compression is needed, but that compression \emph{determines structure}. Given a query space, the minimal compressed representation is unique.

This connects information theory to theory choice: the ``best'' theory is not a matter of taste but of compression. The theorem makes this precise.

\subsection{Objection 7: ``Anti-pluralism is too strong''}

\textbf{Objection:} ``Theorem 4.1 says pluralism is incoherent. But scientists routinely use multiple equivalent formulations.''

\textbf{Response:} The theorem distinguishes:

\begin{itemize}
\item \textbf{Notational pluralism:} Different presentations of the same theory (allowed)
\item \textbf{Structural pluralism:} Different minimal structures for the same queries (impossible)
\end{itemize}

Scientists use multiple \emph{notations} (Lagrangian vs Hamiltonian mechanics), not multiple \emph{structures}. The underlying theory is unique; the presentations vary.

\subsection{Objection 8: ``This doesn't help practitioners''}

\textbf{Objection:} ``Proving uniqueness doesn't help scientists find theories. This is pure philosophy.''

\textbf{Response:} The result is constructive. Theorem 1.3 shows that the minimal theory is computable from the query space. This provides:

\begin{enumerate}
\item \textbf{Search guidance:} Look for the minimal structure that answers all queries
\item \textbf{Termination criterion:} Stop when no further compression is possible
\item \textbf{Validation:} Check that proposed theories are minimal (no redundant structure)
\end{enumerate}

The uniqueness result is not just philosophical---it provides algorithmic guidance for theory construction.

