\section{Uniqueness Theorems}\label{uniqueness-theorems}

\subsection{The Unique Minimal Theory}\label{the-unique-minimal-theory}

\begin{theorem}[Unique Minimal Theory]\label{thm:unique-minimal-theory}
Let $D$ be a finite domain with query space $\text{Queries}(D)$. There exists
exactly one minimal theory $T^* : \text{Queries}(D) \to \text{Answers}$ up to
isomorphism.

Any other theory $T'$ satisfies exactly one of:
\begin{enumerate}
\item $T'$ fails to answer some query in $\text{Queries}(D)$ (incomplete), or
\item $T'$ contains redundant structure not required by any query (non-minimal)
\end{enumerate}
\end{theorem}

\textbf{Definition (Theory Isomorphism).} Two theories $T_1, T_2$ are \emph{isomorphic}
($T_1 \cong T_2$) if there exists a bijection $\phi: T_1 \to T_2$ such that for all
$q \in \text{Queries}(D)$: $T_1(q) = T_2(q)$. Isomorphism is query-answer equivalence.

\textbf{Lemma (Equivalence Implies Structural Isomorphism).} For minimal theories,
query-answer equivalence implies structural isomorphism.

\begin{proof}
Let $T_1, T_2$ be minimal with identical query-answer behavior. Suppose $T_1$ contains
structure $c$ with no correspondent in $T_2$. Since $T_2$ answers all queries without $c$,
component $c$ is not required by any query. But then $T_1$ is not minimal---contradiction.
Therefore every component in $T_1$ has a correspondent in $T_2$ and vice versa.
\end{proof}

\begin{proof}[Proof of Theorem~\ref{thm:unique-minimal-theory}]
We prove this by constructing $T^*$ and showing any other minimal theory is isomorphic.

\textbf{Construction:} Define $T^*$ as follows. For each query $q \in \text{Queries}(D)$:
\begin{enumerate}
\item Determine the minimal information $I_q$ from the implementation required to answer $q$
\item Add $I_q$ to $T^*$ if not derivable from existing structure
\item Derive answer to $q$ from $T^*$
\end{enumerate}

This produces a theory where every component is required by some query.

\textbf{Uniqueness:} Suppose $T_1$ and $T_2$ are both minimal and complete.
By the lemma, query-answer equivalence implies structural isomorphism. Both answer
all queries identically (completeness). By minimality, neither contains structure
beyond what queries require. Therefore $T_1 \cong T_2$.
\end{proof}

\textbf{Interpretation:} The minimal theory is not a choice or convention. It is 
determined by the query space. Scientists who ask the same questions will converge to 
isomorphic theories.

\subsection{Compression Necessity}\label{compression-necessity}

\begin{theorem}[Compression Necessity]\label{thm:compression-necessity}
For any domain $D$ with infinite query space $|\text{Queries}(D)| = \infty$, any 
complete theory $T$ satisfies:
$$|T| < |\text{Implementation}|$$
Direct lookup tables are uncomputable.
\end{theorem}

\begin{proof}
Let $I$ be the complete implementation with state space $\Sigma$. If $|\text{Queries}(D)| = \infty$, 
a lookup table storing all query-answer pairs requires infinite storage.

Any computable theory must compress this infinite space into finite structure. 
Therefore $|T| < |I|$ is necessary for computability.

\textbf{Example:} For a dataset in $\mathbb{R}^d$, queries include:
\begin{itemize}
\item "What is the covariance between attributes $i$ and $j$?" (${d \choose 2}$ queries)
\item "What is the correlation?" (another ${d \choose 2}$ queries)
\item "What is the leverage of point $k$?" ($n$ queries)
\item Infinitely many queries about linear combinations, projections, etc.
\end{itemize}

Storing all answers is impossible. The theory compresses via the covariance matrix 
($d^2$ parameters), from which all queries are computable.
\end{proof}

\textbf{Interpretation:} Compression is not optional. Theories must compress to be 
computable.

\subsection{Computability from Queries}\label{computability-from-queries}

\begin{theorem}[Computability from Queries]\label{thm:computability-from-queries}
The minimal theory $T^*$ is a computable function of the query space:
$$T^* = f(\text{Queries}(D))$$
where $f$ is algorithmic.
\end{theorem}

\begin{proof}
We construct $f$ explicitly:

\textbf{Algorithm $f$:}
\begin{enumerate}
\item Input: $Q = \text{Queries}(D)$
\item Initialize $T = \emptyset$
\item For each query $q \in Q$:
   \begin{enumerate}
   \item Determine minimal information $I_q$ required to answer $q$
   \item If $I_q$ not derivable from $T$, add $I_q$ to $T$
   \end{enumerate}
\item Output: $T$
\end{enumerate}

This algorithm terminates because:
\begin{itemize}
\item Each query requires finite information
\item Adding information is monotone (never remove)
\item Finite domain implies finite minimal theory
\end{itemize}

The output is minimal by construction: every component added is required by some query.
\end{proof}

\textbf{Corollary 3.1.} Theory discovery is mechanical extraction, not creative interpretation.

\begin{proof}
Given domain $D$ and query space $\text{Queries}(D)$, the minimal theory is determined 
by algorithm $f$. No ambiguity exists.
\end{proof}

\textbf{Interpretation:} Discovering the Standard Model is not insight---it is 
running algorithm $f$ on particle physics queries. Newton did not invent gravity; he 
extracted the minimal theory answering mechanical queries.

\begin{center}\rule{0.5\linewidth}{0.5pt}\end{center}

