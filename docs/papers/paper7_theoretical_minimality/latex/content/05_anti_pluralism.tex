\section{Anti-Pluralism Results}\label{anti-pluralism}

\subsection{Incoherence of Pluralism}\label{incoherence-of-pluralism}

\begin{theorem}[Incoherence of Pluralism]\label{thm:anti-pluralism}
For any domain $D$ with query space $\text{Queries}(D)$, there cannot exist 
multiple distinct minimal theories $T_1, T_2$ that are not isomorphic.
\end{theorem}

\begin{proof}
Suppose $T_1$ and $T_2$ are both minimal and complete, but not isomorphic.

Since both are complete, they answer all queries identically. 

Since they are not isomorphic, there exists structure in $T_1$ with no 
corresponding structure in $T_2$, or vice versa.

Without loss of generality, suppose $T_1$ contains component $c$ with no 
correspondent in $T_2$. 

Since $T_2$ answers all queries without $c$, component $c$ is not required by 
any query.

But then $T_1$ is not minimal, contradicting our assumption.
\end{proof}

\textbf{Corollary 4.1 (Convention vs Discovery).} Isomorphic theories are mere 
relabelings. If $T_1 \cong T_2$, they differ only in notation, not content.

\begin{proof}
Isomorphism $\phi: T_1 \to T_2$ preserves all structural relationships. Every 
query answered by $T_1$ is answered identically (up to relabeling) by $T_2$.

The theories contain the same information in different notation.
\end{proof}

\textbf{Interpretation:} When physicists debate equivalent formulations of quantum 
mechanics (Heisenberg vs Schrödinger pictures), they are arguing about notation, 
not discovering different theories. The minimal theory answering quantum mechanical 
queries is unique; the formulations are isomorphic.

\subsection{Philosophical Implications}\label{philosophical-implications-pluralism}

\textbf{Rejection of Epistemic Pluralism:} Pluralism claims multiple equally valid 
theories can explain the same phenomena. Our result shows this is impossible for 
minimal theories.

\textbf{Underdetermination of Theory by Data:} Quine argued observations 
underdetermine theory choice. Our result resolves this: observations determine 
queries, queries determine minimal theory uniquely.

\textbf{Scientific Realism:} Theories converge not by social consensus but by 
computational necessity. The minimal theory is real structure, not convention.

\textbf{Theory Choice:} Occam's razor is not aesthetic preference---it is algorithmic 
requirement. Simpler theories are not "more beautiful," they are the unique minimal 
compression.

\textbf{Theorem 4.2 (Learnability).} The minimal theory $T^*$ is learnable from 
finite query samples.

\begin{proof}
By Theorem~\ref{thm:computability-from-queries}, $T^* = f(\text{Queries}(D))$.

For finite $D$, a finite sample of queries suffices to determine all structure 
required by those queries.

Additional queries either confirm existing structure or add new required components.
\end{proof}

\begin{center}\rule{0.5\linewidth}{0.5pt}\end{center}

