\section{Related Work}\label{related-work}

\textbf{Philosophy of Science:} Quine (1951) argued observations underdetermine theory.
We prove the opposite: queries determine minimal theory uniquely. Kuhn (1962) described
theory choice as paradigm shifts; we show it is algorithmic extraction. Popper (1959)
emphasized falsifiability; we prove constructive uniqueness.

\textbf{Kolmogorov Complexity:} Solomonoff~\cite{solomonoff1964formal} and Kolmogorov~\cite{kolmogorov1965three} independently developed algorithmic information theory, establishing that the complexity of an object is the length of its shortest description. Our minimal theories are the ``shortest descriptions'' for query spaces: the minimal structure sufficient to answer all queries. The connection is precise---minimal theories are Kolmogorov-optimal representations of the domain's semantic content.

\textbf{Generative Complexity:} Heering~\cite{heering2015generative,heering2003software} applied Kolmogorov complexity to software, defining \emph{generative complexity} as the length of the shortest generator for a program family. Our uniqueness theorem (Theorem~\ref{thm:unique-minimal-theory}) extends this: minimal theories are unique shortest generators for query spaces. Where Heering's work remained largely theoretical (Kolmogorov complexity is uncomputable), we provide constructive algorithms (Section~\ref{axis-derivation-algorithm}) that compute minimal theories from domain specifications.

\textbf{Minimum Description Length:} Rissanen~\cite{rissanen1978mdl} established the MDL principle: optimal models minimize total description length. Grünwald~\cite{gruenwald2007mdl} proved MDL-optimal models are unique under mild conditions. Our uniqueness theorem is the epistemic analogue: minimal theories are unique under the constraint of query coverage. The parallel is exact---MDL optimizes description length; we optimize parameter count while maintaining coverage.

\textbf{Learning Theory:} Valiant (1984) established PAC learning. Our learnability
results (Theorem 7.3) show minimal theories have efficient sample complexity.

\textbf{Model Selection:} Akaike (1974) and Schwarz (1978) developed information
criteria for model selection. Our uniqueness result provides theoretical foundation:
the minimal theory is not chosen but determined. Where AIC/BIC provide heuristics for selecting among candidate models, we prove the minimal model is \emph{unique}---selection becomes extraction.

\textbf{Scientific Realism:} Putnam (1975) and Boyd (1984) defended convergence of
scientific theories. Our Theorem~\ref{thm:unique-minimal-theory} formalizes this:
theories converge because minimal structure is unique.

\textbf{Occam's Razor:} Philosophical tradition favoring simpler explanations. We prove
it is computational necessity, not aesthetic preference. The razor is not a heuristic but a theorem: minimal theories are unique, so ``preferring'' simplicity is recognizing that the alternative is redundancy.

\begin{center}\rule{0.5\linewidth}{0.5pt}\end{center}

