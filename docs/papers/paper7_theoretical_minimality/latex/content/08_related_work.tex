\section{Related Work}\label{related-work}

\textbf{Philosophy of Science:} Quine (1951) argued observations underdetermine theory. 
We prove the opposite: queries determine minimal theory uniquely. Kuhn (1962) described 
theory choice as paradigm shifts; we show it is algorithmic extraction. Popper (1959) 
emphasized falsifiability; we prove constructive uniqueness.

\textbf{Kolmogorov Complexity:} Solomonoff (1964), Kolmogorov (1965), and Chaitin (1966) 
developed algorithmic information theory. Our minimal theories are related but defined 
by query coverage rather than description length.

\textbf{Learning Theory:} Valiant (1984) established PAC learning. Our learnability 
results (Theorem 7.3) show minimal theories have efficient sample complexity.

\textbf{Model Selection:} Akaike (1974) and Schwarz (1978) developed information 
criteria for model selection. Our uniqueness result provides theoretical foundation: 
the minimal theory is not chosen but determined.

\textbf{Scientific Realism:} Putnam (1975) and Boyd (1984) defended convergence of 
scientific theories. Our Theorem~\ref{thm:unique-minimal-theory} formalizes this: 
theories converge because minimal structure is unique.

\textbf{Occam's Razor:} Philosophical tradition favoring simpler explanations. We prove 
it is computational necessity, not aesthetic preference.

\begin{center}\rule{0.5\linewidth}{0.5pt}\end{center}

