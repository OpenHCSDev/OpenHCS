\section{Conclusion}\label{sec:conclusion}
%==============================================================================

We have provided the first formal foundations for the Single Source of Truth principle. The key contributions are:

\textbf{1. Formal Definition:} SSOT is defined as DOF = 1, where DOF (Degrees of Freedom) counts independent encoding locations for a fact. This definition is derived from the structure of the problem, not chosen arbitrarily.

\textbf{2. Language Requirements:} We prove that SSOT for structural facts requires (1) definition-time hooks AND (2) introspectable derivation. Both are necessary; both together are sufficient. This is an if-and-only-if theorem.

\textbf{3. Language Evaluation:} Among mainstream languages, only Python satisfies both requirements. CLOS and Smalltalk also satisfy them but are not mainstream. This is proved by exhaustive evaluation.

\textbf{4. Complexity Bounds:} SSOT achieves $O(1)$ modification complexity; non-SSOT requires $\Omega(n)$. The gap is unbounded: for any constant $k$, there exists a codebase size where SSOT provides at least $k\times$ reduction.

\textbf{5. Empirical Validation:} 13 case studies from OpenHCS (45K LoC) demonstrate a mean 14.2$\times$ DOF reduction, with a maximum of 47$\times$ (PR \#44: hasattr migration).

\textbf{Implications:}

\begin{enumerate}
\item \textbf{For practitioners:} If SSOT for structural facts is required, Python (or CLOS/Smalltalk) is necessary. Other mainstream languages cannot achieve SSOT within the language.

\item \textbf{For language designers:} Definition-time hooks and introspection should be considered if DRY is a design goal. Their absence is a deliberate choice with consequences.

\item \textbf{For researchers:} Software engineering principles can be formalized and machine-checked. This paper demonstrates the methodology.
\end{enumerate}

\textbf{Limitations:}
\begin{itemize}
\tightlist
\item Results apply to \emph{structural} facts. Configuration values and runtime state have different characteristics.
\item Empirical validation is from a single codebase. Replication in other domains would strengthen the findings.
\item The complexity bounds are asymptotic. Small codebases may not benefit significantly.
\end{itemize}

\textbf{Future Work:}
\begin{itemize}
\tightlist
\item Extend the formalization to non-structural facts
\item Develop automated DOF measurement tools
\item Study the relationship between DOF and other software quality metrics
\item Investigate SSOT in multi-language systems
\end{itemize}

\textbf{Connection to Leverage Framework:}

SSOT achieves \emph{infinite leverage} in the framework of the companion paper on leverage-driven architecture:
\[
L(\text{SSOT}) = \frac{|\text{Derivations}|}{1} \to \infty
\]

A single source derives arbitrarily many facts. This is the theoretical maximum---no architecture can exceed infinite leverage. The leverage framework provides a unified view: this paper (SSOT) and the companion paper on typing discipline selection are both instances of leverage maximization. The metatheorem---``maximize leverage''---subsumes both results.

All results are machine-checked in Lean 4 with zero \texttt{sorry} placeholders. The proofs are available at \texttt{proofs/ssot/}.

\bibliographystyle{plain}
\bibliography{references}

%==============================================================================
\appendix
%==============================================================================

