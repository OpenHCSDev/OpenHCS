\section{Realizability Requirements for Computational Systems}\label{sec:requirements}
%==============================================================================

We now derive the capabilities necessary and sufficient for computational systems to achieve DOF = 1 (optimal encoding rate). This section answers: \emph{What must a computational system provide for DOF = 1 to be realizable?}

The requirements are information-theoretic necessities, not implementation preferences. Programming languages are one domain where these requirements apply; distributed databases and configuration systems are others.

The requirements are derived from the DOF = 1 optimality condition (Theorem~\ref{thm:dof-optimal}). The proofs establish logical necessity.

\subsection{The Foundational Axiom}\label{sec:axiom}

The derivation rests on one axiom, which follows from how computational systems with declarative structures work:

\begin{axiom}[Structural Fixation]\label{axiom:fixation}
Structural facts are fixed at definition time. After a declarative structure (class, schema, resource) is defined, its structural properties cannot be retroactively changed.
\end{axiom}

This axiom applies across computational domains:

\textbf{Programming languages:}
\begin{itemize}
\tightlist
\item Once \texttt{class Foo extends Bar} is defined, \texttt{Foo}'s parent cannot become \texttt{Baz}
\item Once \texttt{def process(self, x: int)} is defined, the signature cannot retroactively become \texttt{(self, x: str)}
\item Once \texttt{trait Handler} is implemented for \texttt{PNGDecoder}, that relationship is permanent
\end{itemize}

\textbf{Databases:}
\begin{itemize}
\tightlist
\item Once \texttt{CREATE TABLE users (id INT, name TEXT)} executes, the table structure is fixed
\item Altering the schema requires explicit \texttt{ALTER TABLE} (not retroactive modification)
\item Foreign key constraints are fixed at creation time
\end{itemize}

\textbf{Configuration systems:}
\begin{itemize}
\tightlist
\item Once \texttt{resource "aws\_instance" "app"} is declared, the resource type is fixed
\item The dependency graph is determined at declaration time
\end{itemize}

Systems that allow modification (Python's \texttt{\_\_bases\_\_}, \texttt{ALTER TABLE}) are modifying \emph{future} behavior, not \emph{past} structure. The fact that ``table X was created with schema S at time T'' is fixed at the moment of creation.

\textbf{All subsequent theorems are logical consequences of this axiom.} Rejecting the axiom requires demonstrating a computational system where structural facts can be retroactively modified---which does not exist.

\subsection{The Timing Constraint}\label{sec:timing}

The key insight is that structural facts have a \emph{timing constraint}. Unlike configuration values (which can be changed at any time), structural facts are fixed at specific moments:

\begin{definition}[Structural Timing]\label{def:structural-timing}
A structural fact $F$ (class existence, inheritance relationship, method signature) is \emph{fixed} when its defining construct is executed. After that point, the structure cannot be retroactively modified.
\end{definition}

In Python, classes are defined when the \texttt{class} statement executes:

\begin{verbatim}
class Detector(Processor):  # Structure fixed HERE
    def detect(self, img): ...

# After this point, Detector's inheritance cannot be changed
\end{verbatim}

In Java, classes are defined at compile time:

\begin{verbatim}
public class Detector extends Processor {  // Structure fixed at COMPILE TIME
    public void detect(Image img) { ... }
}
\end{verbatim}

\textbf{Critical Distinction: Compile-Time vs. Definition-Time}

These terms are often confused. We define them precisely:

\begin{definition}[Compile-Time]\label{def:compile-time}
\emph{Compile-time} is when source code is translated to an executable form (bytecode, machine code). Compile-time occurs \emph{before the program runs}.
\end{definition}

\begin{definition}[Definition-Time]\label{def:definition-time}
\emph{Definition-time} is when a class/type definition is \emph{executed}. In Python, this is \emph{at runtime} when the \texttt{class} statement runs. In Java, this is \emph{at compile-time} when \texttt{javac} processes the file.
\end{definition}

The key insight: \textbf{Python's \texttt{class} statement is executable code.} When Python encounters:

\begin{verbatim}
class Foo(Bar):
    x = 1
\end{verbatim}

It \emph{executes} code that:
\begin{enumerate}
\tightlist
\item Creates a new namespace
\item Executes the class body in that namespace
\item Calls the metaclass to create the class object
\item Calls \texttt{\_\_init\_subclass\_\_} on parent classes
\item Binds the name \texttt{Foo} to the new class
\end{enumerate}

This is why Python has ``definition-time hooks''---they execute when the definition runs.

Java's \texttt{class} declaration is \emph{not} executable---it is a static declaration processed by the compiler. No user code can hook into this process.

The timing constraint has profound implications for derivation:

\begin{theorem}[Timing Forces Definition-Time Derivation]\label{thm:timing-forces}
Derivation for structural facts must occur at or before the moment the structure is fixed.
\end{theorem}

\begin{proof}
Let $F$ be a structural fact. Let $t_{\text{fix}}$ be the moment $F$ is fixed. Any derivation $D$ that depends on $F$ must execute at some time $t_D$.

Case 1: $t_D < t_{\text{fix}}$. Then $D$ executes before $F$ is fixed. $D$ cannot derive from $F$ because $F$ does not yet exist.

Case 2: $t_D > t_{\text{fix}}$. Then $D$ executes after $F$ is fixed. $D$ can read $F$ but cannot modify structure derived from $F$---the structure is already fixed.

Case 3: $t_D = t_{\text{fix}}$. Then $D$ executes at the moment $F$ is fixed. $D$ can both read $F$ and modify derived structures before they are fixed.

Therefore, derivation for structural facts must occur at definition time ($t_D = t_{\text{fix}}$).
\end{proof}

\subsection{Requirement 1: Definition-Time Computation}\label{sec:hooks}

\begin{definition}[Definition-Time Computation Capability]\label{def:hook}
A computational system has \emph{definition-time computation capability} iff it can execute arbitrary computation when a declarative structure is \emph{created}, not when it is \emph{used}.
\end{definition}

This requirement has theoretical foundations in metaobject protocols~\cite{kiczales1991art} and computational reflection~\cite{smith1984reflection}. Programming language realizations include Python's metaclasses, CLOS initialization hooks, and Smalltalk's class creation protocol.

\textbf{Examples across domains:}

\textbf{Programming languages (Python):}

\begin{center}
\begin{tabular}{lp{8cm}}
\toprule
\textbf{Hook} & \textbf{When it executes} \\
\midrule
\texttt{\_\_init\_subclass\_\_} & When a subclass is defined \\
Metaclass \texttt{\_\_new\_\_}/\texttt{\_\_init\_\_} & When a class using that metaclass is defined \\
Class decorator & Immediately after class body executes \\
\texttt{\_\_set\_name\_\_} & When a descriptor is assigned to a class attribute \\
\bottomrule
\end{tabular}
\end{center}

\textbf{Example: \texttt{\_\_init\_subclass\_\_} registration}

\begin{verbatim}
class Registry:
    _handlers = {}

    def __init_subclass__(cls, format=None, **kwargs):
        super().__init_subclass__(**kwargs)
        if format:
            Registry._handlers[format] = cls

class PNGHandler(Registry, format="png"):
    pass  # Automatically registered when class is defined

class JPGHandler(Registry, format="jpg"):
    pass  # Automatically registered when class is defined

# Registry._handlers == {"png": PNGHandler, "jpg": JPGHandler}
\end{verbatim}

The registration happens at definition time, not at first use. When the \texttt{class PNGHandler} statement executes, \texttt{\_\_init\_subclass\_\_} runs and adds the handler to the registry.

\begin{theorem}[Definition-Time Computation is Necessary]\label{thm:hooks-necessary}
Achieving DOF = 1 for structural facts requires definition-time computation capability.
\end{theorem}

\begin{proof}
By Theorem~\ref{thm:timing-forces}, derivation for structural facts must occur at definition time. Without definition-time computation, no computation can execute at that moment. Therefore, derivation is impossible. Without derivation, secondary encodings cannot be automatically updated. DOF $>$ 1 is unavoidable.

Contrapositive: If a computational system lacks definition-time computation, DOF = 1 for structural facts is unrealizable.
\end{proof}

\textbf{Computational systems lacking this capability:}

\textbf{Programming languages:}
\begin{itemize}
\tightlist
\item \textbf{Java}: Annotations are metadata, not executable computation. They are processed by external tools (annotation processors), not by the runtime at class definition.
\item \textbf{C++}: Templates expand at compile time but do not execute arbitrary code. SFINAE and \texttt{constexpr if} select branches, not execute computation.
\item \textbf{Go}: No hook mechanism. Interfaces are implicit. No code runs at type definition.
\item \textbf{Rust}: Procedural macros run at compile time but are opaque at runtime (see Theorem~\ref{thm:opaque-expansion}).
\end{itemize}

\textbf{Databases (most systems):}
\begin{itemize}
\tightlist
\item Traditional SQL databases do not execute arbitrary user code at schema definition time
\item Triggers execute on \emph{data} operations, not schema operations
\item PostgreSQL event triggers are a partial exception but limited in scope
\end{itemize}

\subsection{Requirement 2: Introspectable Derivation Results}\label{sec:introspection}

Definition-time computation enables derivation. But achieving verifiable DOF = 1 also requires \emph{verification}---the ability to confirm that exactly one independent encoding exists. This requires \emph{computational reflection}---the ability of a system to reason about its own structure~\cite{smith1984reflection}.

\begin{definition}[Introspectable Derivation]\label{def:introspection}
Derivation is \emph{introspectable} iff the program can query:
\begin{enumerate}
\tightlist
\item What structures were derived
\item From which source each derived structure came
\item What the current state of derived structures is
\end{enumerate}
\end{definition}

\textbf{Python's introspection capabilities:}

\begin{center}
\begin{tabular}{lp{7cm}}
\toprule
\textbf{Query} & \textbf{Python Mechanism} \\
\midrule
What subclasses exist? & \texttt{cls.\_\_subclasses\_\_()} \\
What is the inheritance chain? & \texttt{cls.\_\_mro\_\_} \\
What attributes does a class have? & \texttt{dir(cls)}, \texttt{vars(cls)} \\
What type is this object? & \texttt{type(obj)}, \texttt{isinstance(obj, cls)} \\
What methods are abstract? & \texttt{cls.\_\_abstractmethods\_\_} \\
\bottomrule
\end{tabular}
\end{center}

\textbf{Example: Verifying registration completeness}

\begin{verbatim}
def verify_registration():
    """Verify all subclasses are registered."""
    all_subclasses = set(ImageLoader.__subclasses__())
    registered = set(LOADER_REGISTRY.values())

    unregistered = all_subclasses - registered
    if unregistered:
        raise RuntimeError(f"Unregistered loaders: {unregistered}")
\end{verbatim}

This verification is only possible because Python provides \texttt{\_\_subclasses\_\_()}. In systems without this capability, operators cannot enumerate what structures exist.

\begin{theorem}[Introspection is Necessary for Verifiable DOF = 1]\label{thm:introspection-necessary}
Verifying that DOF = 1 holds requires introspection capability.
\end{theorem}

\begin{proof}
Verification of DOF = 1 requires confirming exactly one independent encoding exists. This requires:
\begin{enumerate}
\tightlist
\item Enumerating all locations encoding fact $F$
\item Determining which are independent vs. derived
\item Confirming exactly one is independent
\end{enumerate}

Step (1) requires introspection: the system must query what structures exist and what they encode. Without introspection, the system cannot enumerate encodings. Verification is impossible.

Without verifiable DOF = 1, system operators cannot confirm optimal encoding holds. They must trust implementation correctness without runtime confirmation. Bugs in derivation logic go undetected until coherence violations occur.
\end{proof}

\textbf{Computational systems lacking introspection:}

\textbf{Programming languages:}

\begin{itemize}
\tightlist
\item \textbf{C++}: Cannot query ``what types instantiated template \texttt{Foo<T>}?''
\item \textbf{Rust}: Procedural macro expansion is opaque at runtime. Cannot query what was generated.
\item \textbf{TypeScript}: Types are erased at runtime. Cannot query type relationships.
\item \textbf{Go}: No type registry. Cannot enumerate types implementing an interface.
\end{itemize}

\textbf{Configuration systems:}
\begin{itemize}
\tightlist
\item Many infrastructure-as-code systems lack introspection: cannot query ``what resources derive from this template?''
\item Kubernetes Operators can query derived resources via API server (introspection present)
\end{itemize}

\subsection{Independence of Requirements}\label{sec:independence}

The two requirements---definition-time hooks and introspection---are independent. Neither implies the other.

\begin{theorem}[Requirements are Independent]\label{thm:independence}
\begin{enumerate}
\tightlist
\item A language can have definition-time hooks without introspection
\item A language can have introspection without definition-time hooks
\end{enumerate}
\end{theorem}

\begin{proof}
\textbf{(1) Hooks without introspection:} Rust procedural macros execute at compile time (a form of definition-time hook) but the generated code is opaque at runtime. The program cannot query what the macro generated.

\textbf{(2) Introspection without hooks:} Java provides \texttt{Class.getMethods()}, \texttt{Class.getInterfaces()}, etc. (introspection) but no code executes when a class is defined. Annotations are metadata, not executable hooks.

Therefore, the requirements are independent.
\end{proof}

\subsection{The Realizability Theorem}\label{sec:completeness}

\begin{theorem}[Necessary and Sufficient Realizability Conditions]\label{thm:ssot-iff}
A computational system $S$ can achieve verifiable DOF = 1 for structural facts if and only if:
\begin{enumerate}
\tightlist
\item $S$ provides definition-time computation capability, AND
\item $S$ provides introspectable derivation results
\end{enumerate}
\end{theorem}

\begin{proof}
$(\Rightarrow)$ \textbf{Necessity:} Suppose $S$ achieves verifiable DOF = 1 for structural facts.
\begin{itemize}
\tightlist
\item By Theorem~\ref{thm:hooks-necessary}, $S$ must provide definition-time computation
\item By Theorem~\ref{thm:introspection-necessary}, $S$ must provide introspection
\end{itemize}

$(\Leftarrow)$ \textbf{Sufficiency:} Suppose $S$ provides both capabilities.
\begin{itemize}
\tightlist
\item Definition-time computation enables derivation at the right moment (when structure is fixed)
\item Introspection enables verification that all secondary encodings are derived
\item Therefore, DOF = 1 is achievable: create one source, derive all others, verify completeness
\end{itemize}

The if-and-only-if follows.
\end{proof}

\begin{corollary}[DOF-1-Complete Systems]\label{cor:ssot-complete}
A computational system is \emph{DOF-1-complete} (can achieve optimal encoding rate for structural facts) iff it satisfies both requirements. Otherwise it is \emph{DOF-1-incomplete}.
\end{corollary}

\subsection{The Logical Chain (Summary)}\label{sec:chain}

For clarity, we summarize the complete derivation from axiom to realizability theorem:

\begin{center}
\fbox{\parbox{0.9\textwidth}{
\textbf{Axiom~\ref{axiom:fixation}:} Structural facts are fixed at definition time.

$\downarrow$ (definitional)

\textbf{Theorem~\ref{thm:timing-forces}:} Derivation for structural facts must occur at definition time.

$\downarrow$ (logical necessity)

\textbf{Theorem~\ref{thm:hooks-necessary}:} Definition-time computation is necessary for DOF = 1.

\textbf{Theorem~\ref{thm:introspection-necessary}:} Introspection is necessary for verifiable DOF = 1.

$\downarrow$ (conjunction)

\textbf{Theorem~\ref{thm:ssot-iff}:} A computational system can achieve DOF = 1 iff it has both capabilities.

$\downarrow$ (evaluation)

\textbf{Corollary:} Among programming languages, Python, CLOS, Smalltalk are DOF-1-complete. Java, C++, Rust, Go, TypeScript are not.
}}
\end{center}

\textbf{Every step is machine-checked in Lean 4.} The proofs compile with zero \texttt{sorry} placeholders. Rejecting this chain requires identifying a specific flaw in the axiom, the logic, or the Lean formalization.

\subsection{Concrete Impossibility Demonstration}\label{sec:impossibility}

We now demonstrate \emph{exactly why} DOF-1-incomplete systems cannot achieve DOF = 1 for structural facts. This is not about system quality---it is about fundamental capabilities that certain systems \emph{cannot provide} within their design constraints.

\textbf{The Structural Fact:} ``\texttt{PNGHandler} handles \texttt{.png} files.''

This fact must be encoded in two places:
\begin{enumerate}
\tightlist
\item The class definition (where the handler is defined)
\item The registry/dispatcher (where format→handler mapping lives)
\end{enumerate}

\textbf{Python achieves DOF = 1:}

\begin{verbatim}
class ImageHandler:
    _registry = {}

    def __init_subclass__(cls, format=None, **kwargs):
        super().__init_subclass__(**kwargs)
        if format:
            ImageHandler._registry[format] = cls  # DERIVED

class PNGHandler(ImageHandler, format="png"):  # SOURCE
    def load(self, path): ...
\end{verbatim}

DOF = 1. The \texttt{format="png"} in the class definition is the \emph{single source}. The registry entry is \emph{derived} automatically by \texttt{\_\_init\_subclass\_\_}. Adding a new handler requires changing exactly one location.

\textbf{Java cannot achieve DOF = 1:}

\begin{verbatim}
// File 1: PNGHandler.java
@Handler(format = "png")  // Annotation is METADATA, not executable
public class PNGHandler implements ImageHandler {
    public BufferedImage load(String path) { ... }
}

// File 2: HandlerRegistry.java (SEPARATE SOURCE!)
public class HandlerRegistry {
    static {
        register("png", PNGHandler.class);  // Must be maintained manually
        register("jpg", JPGHandler.class);
        // Forgot to add TIFFHandler? Runtime error.
    }
}
\end{verbatim}

DOF = 2. The \texttt{@Handler(format = "png")} annotation is \emph{data}, not code. It does not execute when the class is defined. The registry must be maintained separately.

\begin{theorem}[Generated Files Are Independent Encodings]\label{thm:generated-second}
A generated source file constitutes an independent encoding, not a derivation. Therefore, code generation does not achieve DOF = 1.
\end{theorem}

\begin{proof}
Let $F$ be a structural fact (e.g., ``PNGHandler handles .png files'').

Let $E_1$ be the annotation: \texttt{@Handler(format="png")} on \texttt{PNGHandler.java}.

Let $E_2$ be the generated file: \texttt{HandlerRegistry.java} containing \texttt{register("png", PNGHandler.class)}.

By Definition~\ref{def:dof}, $E_1$ and $E_2$ are both encodings of $F$ iff modifying either can change the system's behavior regarding $F$.

Test: If we delete or modify \texttt{HandlerRegistry.java}, does the system's behavior change? \textbf{Yes}---the handler will not be registered.

Test: If we modify the annotation, does the system's behavior change? \textbf{Yes}---the generated file will have different content.

Therefore, $E_1$ and $E_2$ are independent encodings. DOF $= 2$.

Formally: if an artifact $r$ is absent from the system's runtime state (cannot be queried or mutated in-process), then $\text{encodes}(r,F)$ introduces an independent degree of freedom.

The fact that $E_2$ was \emph{generated from} $E_1$ does not make it a derivation, because:
\begin{enumerate}
\tightlist
\item $E_2$ exists as a separate artifact that can be edited, deleted, or fail to generate
\item $E_2$ must be separately compiled/processed
\item The generation process is external to the runtime and can be bypassed
\end{enumerate}

Contrast with Python, where the registry entry exists only in memory, created by the class statement itself. There is no second file. DOF $= 1$.
\end{proof}

\textbf{Why Rust proc macros don't help:}

\begin{theorem}[Opaque Expansion Prevents Verification]\label{thm:opaque-expansion}
If macro/template expansion is opaque at runtime, DOF = 1 cannot be verified.
\end{theorem}

\begin{proof}
Verification of DOF = 1 requires answering: ``Is every encoding of $F$ derived from the single source?''

This requires enumerating all encodings. If expansion is opaque, the system cannot query what was generated.

In Rust, after \texttt{\#[derive(Handler)]} expands, the program cannot ask ``what did this macro generate?'' The expansion is compiled into the binary but not introspectable.

Without introspection, the system cannot verify DOF $= 1$. Optimal encoding may hold but cannot be confirmed.
\end{proof}

\textbf{The Gap is Fundamental:}

The distinction is not about syntax quality or language preference. The distinction is about fundamental capabilities:

\begin{itemize}
\tightlist
\item \textbf{Python:} Class definition \emph{executes code} that creates derived structures \emph{in runtime state}
\item \textbf{Java:} Class definition \emph{produces metadata} that external tools process into \emph{separate files}
\item \textbf{Rust:} Macro expansion \emph{is invisible at runtime}---verification impossible
\end{itemize}

This is a design choice with permanent consequences for realizability. No amount of engineering in Java can make the registry \emph{derived from} the class definition in the DOF = 1 sense, because Java's execution model provides no mechanism for arbitrary code execution at class definition time.

%==============================================================================
