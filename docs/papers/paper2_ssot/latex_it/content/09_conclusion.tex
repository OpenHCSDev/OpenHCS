\section{Conclusion}\label{sec:conclusion}
%==============================================================================

\subsection*{Methodology and Disclosure}

\textbf{Role of LLMs in this work.} This paper was developed through
human-AI collaboration. The author provided the core intuitions (the
DOF formalization, the DEF+INTRO conjecture, the language evaluation
criteria), while large language models (Claude, GPT-4) served as
implementation partners for drafting proofs, formalizing definitions,
and generating LaTeX.

The Lean 4 proofs were iteratively developed: the author specified
theorems to prove, the LLM proposed proof strategies, and the Lean
compiler verified correctness. This is epistemically sound: a Lean proof
that compiles is correct regardless of generation method. The proofs are
\emph{costly signals} (per the companion paper on credibility) whose
validity is independent of their provenance.

\textbf{What the author contributed:} The DOF = 1 formalization of SSOT,
the DEF+INTRO language requirements, the claim that Python uniquely
satisfies these among mainstream languages, the OpenHCS case studies,
and the complexity bounds.

\textbf{What LLMs contributed:} LaTeX drafting, Lean tactic exploration,
prose refinement, and literature search assistance.

Transparency about this methodology reflects our belief that the
contribution is the insight and the verified proof, not the typing
labor.

\begin{center}\rule{0.5\linewidth}{0.5pt}\end{center}

We have established the first information-theoretic foundations for optimal encoding under coherence constraints. The key contributions are:

\textbf{1. Extension of Source Coding Theory:} We extend classical source coding to \emph{interactive encoding systems}---systems where encodings can be modified and must remain coherent across modifications. DOF (Degrees of Freedom) formalizes encoding rate as the count of independent encoding locations for a fact.

\textbf{2. Optimal Rate Uniqueness:} We prove that DOF = 1 is the \textbf{unique} optimal encoding rate guaranteeing coherence (Theorem~\ref{thm:ssot-unique}). Any system with DOF $> 1$ permits incoherent states; DOF = 0 fails to represent the fact. This uniqueness is information-theoretic necessity, not design choice.

\textbf{3. Rate-Complexity Tradeoffs:} We establish fundamental tradeoffs analogous to rate-distortion theory: DOF = 1 achieves $O(1)$ modification complexity; DOF $> 1$ requires $\Omega(n)$. The gap is unbounded---for any constant $k$, there exists an encoding system size where DOF = 1 provides at least $k\times$ reduction (Theorem~\ref{thm:unbounded-gap}).

\textbf{4. Resolution Impossibility:} We prove an impossibility theorem (Theorem~\ref{thm:oracle-arbitrary}) analogous to zero-error capacity: without coherence guarantees, no resolution procedure is information-theoretically justified. Multiple independent encodings create irresolvable ambiguity.

\textbf{5. Realizability Requirements:} For computational systems, we prove that DOF = 1 realizability requires (1) definition-time computation AND (2) introspectable derivation (Theorem~\ref{thm:ssot-iff}). Both are necessary; both together are sufficient. This is an if-and-only-if characterization.

\textbf{6. Mathematical Necessity:} The uniqueness theorem (Theorem~\ref{thm:ssot-unique}) establishes that DOF=1 is the unique minimal encoding rate: $|\{r : \text{optimal}(r)\}| = 1$. This singleton solution space eliminates design freedom. Given coherence as a requirement, the mathematics forces DOF = 1.

\textbf{Corollary instantiations.} We include a programming-language instantiation and a worked case study as corollaries of the realizability theorem. These illustrate the abstract results without being used in the proofs.

\textbf{Implications:}

\begin{enumerate}
\item \textbf{For encoding system designers:} If coherence guarantees are required, the system must provide automatic derivation mechanisms; otherwise coherence scales as $\Omega(n)$ with the number of independent encodings.

\item \textbf{For information theorists:} Classical source coding extends to interactive systems with modification constraints. The coherence requirement creates rate-complexity tradeoffs analogous to rate-distortion tradeoffs, with a unique optimal rate.

\item \textbf{For formal methods researchers:} The results can be formalized in a proof assistant; the Lean proofs show theorems and model definitions are mechanizable.
\end{enumerate}

\textbf{Limitations:}
\begin{itemize}
\tightlist
\item Results apply primarily to facts with modification constraints. Streaming data and high-frequency updates have different characteristics.
\item The complexity bounds are asymptotic. For small encoding systems (DOF $< 5$), the asymptotic gap is numerically small.
\item Computational realization examples are primarily from software systems. The theory is general, but database and configuration system case studies are limited to canonical examples.
\item Realizability requirements focus on computational systems. Physical and biological encoding systems require separate analysis.
\end{itemize}

\textbf{Future Work:}
\begin{itemize}
\tightlist
\item Extend the encoding theory to probabilistic coherence (soft constraints, approximate agreement)
\item Develop automated DOF measurement tools for multiple computational domains (code analysis, schema analysis, configuration analysis)
\item Study the relationship between DOF and other system quality metrics (reliability, maintainability, performance)
\item Investigate DOF = 1 realizability in distributed systems with network partitions
\item Characterize the information-theoretic limits of compile-time vs. runtime coherence mechanisms
\end{itemize}

\subsection{Artifacts}\label{sec:data-availability}

The Lean 4 formalization is included as supplementary material~\cite{openhcsLeanProofs}. The OpenHCS case study and associated code references are provided for the worked instantiation (Section~\ref{sec:empirical}).
