\section{Related Work}\label{sec:related}
%==============================================================================

This section surveys related work across five areas: source coding under modification constraints, distributed systems consistency, computational reflection, software engineering principles, and formal methods.

\subsection{Source Coding Under Modification Constraints}\label{sec:related-source-coding}

Our work extends classical source coding to \emph{interactive encoding systems}---systems where encodings can be modified and must remain coherent across modifications. This connects to several established IT areas.

\textbf{Multi-Version Coding.} Rashmi et al.~\cite{rashmi2015multiversion} formalize consistent distributed storage where multiple versions of data must be accessible while maintaining consistency guarantees. Their framework addresses a key question: what is the storage cost of ensuring that any $c$ servers can decode the latest common version? They prove an ``inevitable price, in terms of storage cost, to ensure consistency.''

Our DOF = 1 theorem is analogous: we prove the \emph{encoding rate} cost of ensuring coherence under modification. Where multi-version coding trades storage for consistency across versions, we trade encoding rate for coherence across locations.

\textbf{Write-Once Memory Codes.} Rivest and Shamir~\cite{rivest1982wom} introduced WOM codes for storage media where bits can only transition $0 \to 1$. Despite this irreversibility constraint, clever coding achieves capacity $\log_2(t+1)$ for $t$ writes---more than the naive $1$ bit.

Our structural facts have an analogous irreversibility: once defined, structure is fixed. The parallel:
\begin{itemize}
\tightlist
\item \textbf{WOM:} Physical irreversibility (bits only increase) $\Rightarrow$ coding schemes maximize information per cell
\item \textbf{DOF = 1:} Structural irreversibility (definition is permanent) $\Rightarrow$ derivation schemes minimize independent encodings
\end{itemize}
Wolf~\cite{wolf1984wom} extended WOM capacity results; our realizability theorem (Theorem~\ref{thm:ssot-iff}) characterizes what encoding systems can achieve DOF = 1 under structural constraints.

\textbf{Classical Source Coding.} Shannon~\cite{shannon1948mathematical} established source coding theory for static data. Slepian and Wolf~\cite{slepian1973noiseless} extended to distributed sources with correlated side information, proving that joint encoding of $(X, Y)$ can achieve rate $H(X|Y)$ for $X$ when $Y$ is available at the decoder.

Our provenance observability requirement (Section~\ref{sec:provenance-observability}) is the encoding-system analog: the decoder (verification procedure) has ``side information'' about the derivation structure, enabling verification of DOF = 1 without examining all locations independently.

\textbf{Rate-Distortion Theory.} Cover and Thomas~\cite{cover2006elements} formalize the rate-distortion function $R(D)$: the minimum encoding rate to achieve distortion $D$. Our rate-complexity tradeoff (Theorem~\ref{thm:unbounded-gap}) is analogous: encoding rate (DOF) trades against modification complexity. DOF = 1 achieves $O(1)$ complexity; DOF $> 1$ incurs $\Omega(n)$.

\textbf{Interactive Information Theory.} The BIRS workshop~\cite{birs2012interactive} identified interactive information theory as an emerging area combining source coding, channel coding, and directed information. Ma and Ishwar~\cite{ma2011distributed} showed that interaction can reduce rate for function computation. Xiang~\cite{xiang2013interactive} studied interactive schemes including feedback channels.

Our framework extends this to \emph{storage} rather than communication: encoding systems where the encoding itself is modified over time, requiring coherence maintenance.

\textbf{Minimum Description Length.} Rissanen~\cite{rissanen1978mdl} established MDL: the optimal model minimizes total description length (model + data given model). Grünwald~\cite{gruenwald2007mdl} proved uniqueness of MDL-optimal representations.

DOF = 1 is the MDL-optimal encoding for redundant facts: the single source is the model; derived locations have zero marginal description length (fully determined by source). Additional independent encodings add description length without reducing uncertainty---pure overhead. Our Theorem~\ref{thm:dof-optimal} establishes analogous uniqueness for encoding systems under modification constraints.

\paragraph{Closest prior work and novelty.}
The closest IT lineage is multi-version coding and zero-error/interactive source coding. These settings address consistency or decoding with side information, but they do not model \emph{modifiable} encodings with a coherence constraint over time. Our contribution is a formal encoding model with explicit modification operations, a coherence capacity theorem (unique rate for guaranteed coherence), an iff realizability characterization, and tight rate--complexity bounds.

\subsection{Distributed Systems Consistency}\label{sec:related-distributed}

We give formal encoding-theoretic versions of CAP and FLP in Section~\ref{sec:cap-flp}. The connection is structural: CAP corresponds to the impossibility of coherence when replicated encodings remain independently updatable, and FLP corresponds to the impossibility of truth-preserving resolution in incoherent states without side information. Consensus protocols (Paxos~\cite{lamport1998paxos}, Raft~\cite{ongaro2014raft}) operationalize this by enforcing coordination, which in our model corresponds to derivation (reducing DOF).

\subsection{Computational Reflection and Metaprogramming}\label{sec:related-meta}

\textbf{Metaobject protocols and reflection.} Kiczales et al.~\cite{kiczales1991art} and Smith~\cite{smith1984reflection} provide the classical foundations for systems that can execute code at definition time and introspect their own structure. These mechanisms correspond directly to DEF and INTRO in our realizability theorem, explaining why MOP-equipped languages admit DOF = 1 for structural facts.

\textbf{Generative complexity.} Heering~\cite{heering2015generative,heering2003software} formalizes minimal generators for program families. DOF = 1 systems realize this minimal-generator viewpoint by construction: the single source is the generator and derived locations are generated instances.

\subsection{Software Engineering Principles}\label{sec:related-software}

Classical software-engineering principles such as DRY~\cite{hunt1999pragmatic}, information hiding~\cite{parnas1972criteria}, and code-duplication analyses~\cite{fowler1999refactoring,roy2007survey} motivate coherence and single-source design. Our contribution is not another guideline, but a formal encoding model and theorems that explain when such principles are forced by information constraints. These connections are interpretive; the proofs do not rely on SE assumptions.

\subsection{Formal Methods}\label{sec:related-formal}

Our Lean 4~\cite{demoura2021lean4} formalization follows the tradition of mechanized theory (e.g., Pierce~\cite{pierce2002types}, Winskel~\cite{winskel1993semantics}, CompCert~\cite{leroy2009compcert}), but applies it to an information-theoretic encoding model.

\subsection{Novelty of This Work}\label{sec:novelty}

To our knowledge, this is the first work to:
\begin{enumerate}
\tightlist
\item \textbf{Formalize interactive encoding with modifications}---a model where encodings change over time and coherence is a system property, not a post hoc check.

\item \textbf{Prove a coherence capacity theorem}---DOF = 1 is the unique rate guaranteeing coherence (achievability + converse).

\item \textbf{Give a realizability iff}---causal propagation and provenance observability are necessary and sufficient encoder properties for achieving DOF = 1.

\item \textbf{Establish tight rate--complexity bounds}---$O(1)$ for DOF = 1 vs.\ $\Omega(n)$ for DOF $> 1$, with an unbounded gap.

\item \textbf{Provide machine-checked proofs}---All theorems formalized in Lean 4 with 0 \texttt{sorry} placeholders.
\end{enumerate}

\textbf{Information-theoretic contribution:} We extend classical IT to mutable encoding systems with coherence constraints. The coherence capacity theorem and tight rate--complexity bounds provide the achievability/converse structure; the realizability iff identifies the encoder properties required to attain capacity.

\textbf{Interpretive instantiations:} The abstract requirements instantiate across domains (e.g., programming languages and database systems). These instantiations are corollaries of the core theorems and are presented as examples, not as premises of the proofs.

%==============================================================================
